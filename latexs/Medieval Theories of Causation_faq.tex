\hypertarget{medieval-theories-of-causation}{%
\section{\texorpdfstring{\href{https://plato.stanford.edu/entries/causation-medieval/index.html}{Medieval
Theories of
Causation}}{Medieval Theories of Causation}}\label{medieval-theories-of-causation}}

\hypertarget{what-is-a-state-of-the-soul}{%
\subsection{What is a state of the
soul?}\label{what-is-a-state-of-the-soul}}

\begin{quote}
\emph{summarized\_paragraph} : An example of motion in the wider sense
is an act of the will. It is a change of state of some entity (namely
the mind or soul), but would not have been thought of as local motion by
most medieval thinkers. Thought and will were generally regarded as
immaterial processes. An example is an action of the mind, such as
changing the state of a person's mind. This would be seen as a change in
state of the person's soul or mind.
\end{quote}

\begin{quote}
\emph{avg\_grammar\_rating} : nan\\
\emph{avg\_answerability\_rating} : nan\\
\emph{sum\_yes\_meaningful} : 0\\
\emph{sum\_no\_meaning} : 0\\
\emph{sum\_maybe\_meaning} : 0
\end{quote}

\hypertarget{what-is-another-term-for-the-mind}{%
\subsection{What is another term for the
mind?}\label{what-is-another-term-for-the-mind}}

\begin{quote}
\emph{summarized\_paragraph} : An example of motion in the wider sense
is an act of the will. It is a change of state of some entity (namely
the mind or soul), but would not have been thought of as local motion by
most medieval thinkers. Thought and will were generally regarded as
immaterial processes. An example is an action of the mind, such as
changing the state of a person's mind. This would be seen as a change in
state of the person's soul or mind.
\end{quote}

\begin{quote}
\emph{avg\_grammar\_rating} : nan\\
\emph{avg\_answerability\_rating} : nan\\
\emph{sum\_yes\_meaningful} : 0\\
\emph{sum\_no\_meaning} : 0\\
\emph{sum\_maybe\_meaning} : 0
\end{quote}

\hypertarget{what-do-many-of-the-criticisms-tend-to-a-relational-account-of}{%
\subsection{What do many of the criticisms tend to a relational account
of?}\label{what-do-many-of-the-criticisms-tend-to-a-relational-account-of}}

\begin{quote}
\emph{summarized\_paragraph} : Perception was, throughout the Middle
Ages, a contentious topic, and it was also a topic in which the answers
to strictly causal questions could influence philosophical positions.
The `traditional' view, dating back to Roger Bacon in the mid-thirteenth
century, was that physical objects were known because they caused a
succession of likenesses, or species. This position was attacked by
thinkers such as Henry of Ghent, Peter Olivi, and Duns Scotus.
\textbf{Many of these criticisms tend towards a relational account of
perception, in which -- although species still play a role -- the role
that they play is to be a means by which we know things.}
\end{quote}

\begin{quote}
\emph{avg\_grammar\_rating} : nan\\
\emph{avg\_answerability\_rating} : nan\\
\emph{sum\_yes\_meaningful} : 0\\
\emph{sum\_no\_meaning} : 0\\
\emph{sum\_maybe\_meaning} : 0
\end{quote}

\hypertarget{do-many-of-these-criticisms-tend-to-a-relational-account-of-perception}{%
\subsection{Do many of these criticisms tend to a relational account of
perception?}\label{do-many-of-these-criticisms-tend-to-a-relational-account-of-perception}}

\begin{quote}
\emph{summarized\_paragraph} : Perception was, throughout the Middle
Ages, a contentious topic, and it was also a topic in which the answers
to strictly causal questions could influence philosophical positions.
The `traditional' view, dating back to Roger Bacon in the mid-thirteenth
century, was that physical objects were known because they caused a
succession of likenesses, or species. This position was attacked by
thinkers such as Henry of Ghent, Peter Olivi, and Duns Scotus.
\textbf{Many of these criticisms tend towards a relational account of
perception, in which -- although species still play a role -- the role
that they play is to be a means by which we know things.}
\end{quote}

\begin{quote}
\emph{avg\_grammar\_rating} : nan\\
\emph{avg\_answerability\_rating} : nan\\
\emph{sum\_yes\_meaningful} : 0\\
\emph{sum\_no\_meaning} : 0\\
\emph{sum\_maybe\_meaning} : 0
\end{quote}

\hypertarget{what-kind-of-doubts-are-there-about-motion-and-rest}{%
\subsection{What kind of doubts are there about motion and
rest?}\label{what-kind-of-doubts-are-there-about-motion-and-rest}}

\begin{quote}
\emph{summarized\_paragraph} : Buridan and Oresme are -- in some sense
-- precursors of Galileo. Their causal ontology is still, in important
respects, thoroughly medieval. \textbf{Despite persistent doubts, there
is still something of a distinction between motion and rest, and motion
can only be the result of agency.} Contrast this with Galileo's or --
still more -- Newton's account. Here uniform motion and. rest are
treated on an equal footing, and, consequently, there can be no.
unequivocal distinction betweenmotion and rest.
\end{quote}

\begin{quote}
\emph{avg\_grammar\_rating} : nan\\
\emph{avg\_answerability\_rating} : nan\\
\emph{sum\_yes\_meaningful} : 0\\
\emph{sum\_no\_meaning} : 0\\
\emph{sum\_maybe\_meaning} : 0
\end{quote}

\hypertarget{despite-persistent-doubts-there-is-still-something-of-what-between-motion-and-rest}{%
\subsection{Despite persistent doubts , there is still something of what
between motion and
rest?}\label{despite-persistent-doubts-there-is-still-something-of-what-between-motion-and-rest}}

\begin{quote}
\emph{summarized\_paragraph} : Buridan and Oresme are -- in some sense
-- precursors of Galileo. Their causal ontology is still, in important
respects, thoroughly medieval. \textbf{Despite persistent doubts, there
is still something of a distinction between motion and rest, and motion
can only be the result of agency.} Contrast this with Galileo's or --
still more -- Newton's account. Here uniform motion and. rest are
treated on an equal footing, and, consequently, there can be no.
unequivocal distinction betweenmotion and rest.
\end{quote}

\begin{quote}
\emph{avg\_grammar\_rating} : nan\\
\emph{avg\_answerability\_rating} : nan\\
\emph{sum\_yes\_meaningful} : 0\\
\emph{sum\_no\_meaning} : 0\\
\emph{sum\_maybe\_meaning} : 0
\end{quote}

\hypertarget{what-would-a-person-change-in-the-soul-or-mind}{%
\subsection{What would a person change in the soul or
mind?}\label{what-would-a-person-change-in-the-soul-or-mind}}

\begin{quote}
\emph{summarized\_paragraph} : An example of motion in the wider sense
is an act of the will. It is a change of state of some entity (namely
the mind or soul), but would not have been thought of as local motion by
most medieval thinkers. Thought and will were generally regarded as
immaterial processes. An example is an action of the mind, such as
changing the state of a person's mind. \textbf{This would be seen as a
change in state of the person's soul or mind.}
\end{quote}

\begin{quote}
\emph{avg\_grammar\_rating} : nan\\
\emph{avg\_answerability\_rating} : nan\\
\emph{sum\_yes\_meaningful} : 0\\
\emph{sum\_no\_meaning} : 0\\
\emph{sum\_maybe\_meaning} : 0
\end{quote}

\hypertarget{what-do-stoics-believe-there-is-no-end-to}{%
\subsection{What do stoics believe there is no end
to?}\label{what-do-stoics-believe-there-is-no-end-to}}

\begin{quote}
\emph{summarized\_paragraph} : Aristotle believes that there are
processes in nature which are completed and regulated by a final state,
or end, towards which they tend. As Adams puts it, Aristotle has a much
stronger position on final causality than the Stoics. \textbf{The Stoics
believe that there is no such thing as the end of a process, just the
beginning of a new process.} Adams says that Aristotle's view is wrong,
and that the process of growing a tree, for example, is not the end, but
the beginning.
\end{quote}

\begin{quote}
\emph{avg\_grammar\_rating} : nan\\
\emph{avg\_answerability\_rating} : nan\\
\emph{sum\_yes\_meaningful} : 0\\
\emph{sum\_no\_meaning} : 0\\
\emph{sum\_maybe\_meaning} : 0
\end{quote}

\hypertarget{what-did-ockhams-position-not-have-much-on-his-contemporaries}{%
\subsection{What did ockham's position not have much on his
contemporaries?}\label{what-did-ockhams-position-not-have-much-on-his-contemporaries}}

\begin{quote}
\emph{summarized\_paragraph} : Ockham denies species not on the basis of
empirical evidence, but purely and simply on the based of his razor. If
we deny species, then we can give an account of the phenomena which uses
fewer entities, because species are entities. \textbf{Although this
position of Ockham's did not have much influence on his contemporaries
or followers, it is a good example of how causal reasoning is affected
by tacit ontological assumptions.} The fact that species were seen as
entities, led to an account which tried to do away with species. On the
other hand, action at a distance was, despite its implausibility,
entirely unaffected by the critique.
\end{quote}

\begin{quote}
\emph{avg\_grammar\_rating} : nan\\
\emph{avg\_answerability\_rating} : nan\\
\emph{sum\_yes\_meaningful} : 0\\
\emph{sum\_no\_meaning} : 0\\
\emph{sum\_maybe\_meaning} : 0
\end{quote}

\hypertarget{how-much-information-is-found-in-the-book-on-the-structure-of-science}{%
\subsection{How much information is found in the book on the structure
of
science?}\label{how-much-information-is-found-in-the-book-on-the-structure-of-science}}

\begin{quote}
\emph{summarized\_paragraph} : There is an extensive literature of
medieval commentaries on the Posterior Analytics. However, it cannot be
taken to be automatically relevant to the practice of reasoning in the
Middle Ages. The logical metatheory (that of the syllogism) is far too
restrictive, and the conditions placed on scientific demonstrations are
far too stringent, for it to be a plausible description of very many
actual processes of reasoning, in theMiddle Ages or at any other time.
\textbf{We find in it a great deal of material on the authors' attitudes
to necessity, the structure of science, the relation between various
sciences.}
\end{quote}

\begin{quote}
\emph{avg\_grammar\_rating} : nan\\
\emph{avg\_answerability\_rating} : nan\\
\emph{sum\_yes\_meaningful} : 0\\
\emph{sum\_no\_meaning} : 0\\
\emph{sum\_maybe\_meaning} : 0
\end{quote}

\hypertarget{what-does-he-often-do-about-final-causes}{%
\subsection{What does he often do about final
causes?}\label{what-does-he-often-do-about-final-causes}}

\begin{quote}
\emph{summarized\_paragraph} : The medieval literature is far from
unanimous on these questions. William of Ockham, for example, who wrote
several commentaries on Aristotle's Physics, hardly has a uniform
position. \textbf{He is quite happy with explanations of natural
phenomena by means of efficient causes in general, but he will also
often speak of final causes.} What is unclear is whether the final
causes he speaks of (with varying degrees of strength in different
works) have any explanatory role to play that cannot be reduced to
efficient causality.
\end{quote}

\begin{quote}
\emph{avg\_grammar\_rating} : nan\\
\emph{avg\_answerability\_rating} : nan\\
\emph{sum\_yes\_meaningful} : 0\\
\emph{sum\_no\_meaning} : 0\\
\emph{sum\_maybe\_meaning} : 0
\end{quote}

\hypertarget{what-is-he-happy-with}{%
\subsection{What is he happy with?}\label{what-is-he-happy-with}}

\begin{quote}
\emph{summarized\_paragraph} : The medieval literature is far from
unanimous on these questions. William of Ockham, for example, who wrote
several commentaries on Aristotle's Physics, hardly has a uniform
position. \textbf{He is quite happy with explanations of natural
phenomena by means of efficient causes in general, but he will also
often speak of final causes.} What is unclear is whether the final
causes he speaks of (with varying degrees of strength in different
works) have any explanatory role to play that cannot be reduced to
efficient causality.
\end{quote}

\begin{quote}
\emph{avg\_grammar\_rating} : nan\\
\emph{avg\_answerability\_rating} : nan\\
\emph{sum\_yes\_meaningful} : 0\\
\emph{sum\_no\_meaning} : 0\\
\emph{sum\_maybe\_meaning} : 0
\end{quote}

\hypertarget{is-it-clear-about-the-final-causes-of-the-work-that-he-speaks-of}{%
\subsection{Is it clear about the final causes of the work that he
speaks
of?}\label{is-it-clear-about-the-final-causes-of-the-work-that-he-speaks-of}}

\begin{quote}
\emph{summarized\_paragraph} : The medieval literature is far from
unanimous on these questions. William of Ockham, for example, who wrote
several commentaries on Aristotle's Physics, hardly has a uniform
position. He is quite happy with explanations of natural phenomena by
means of efficient causes in general, but he will also often speak of
final causes. What is unclear is whether the final causes he speaks of
(with varying degrees of strength in different works) have any
explanatory role to play that cannot be reduced to efficient causality.
\end{quote}

\begin{quote}
\emph{avg\_grammar\_rating} : nan\\
\emph{avg\_answerability\_rating} : nan\\
\emph{sum\_yes\_meaningful} : 0\\
\emph{sum\_no\_meaning} : 0\\
\emph{sum\_maybe\_meaning} : 0
\end{quote}

\hypertarget{what-quality-of-the-final-causes-does-he-use-in-different-works}{%
\subsection{What quality of the final causes does he use in different
works?}\label{what-quality-of-the-final-causes-does-he-use-in-different-works}}

\begin{quote}
\emph{summarized\_paragraph} : The medieval literature is far from
unanimous on these questions. William of Ockham, for example, who wrote
several commentaries on Aristotle's Physics, hardly has a uniform
position. He is quite happy with explanations of natural phenomena by
means of efficient causes in general, but he will also often speak of
final causes. What is unclear is whether the final causes he speaks of
(with varying degrees of strength in different works) have any
explanatory role to play that cannot be reduced to efficient causality.
\end{quote}

\begin{quote}
\emph{avg\_grammar\_rating} : nan\\
\emph{avg\_answerability\_rating} : nan\\
\emph{sum\_yes\_meaningful} : 0\\
\emph{sum\_no\_meaning} : 0\\
\emph{sum\_maybe\_meaning} : 0
\end{quote}

\hypertarget{who-says-that-aristotles-view-is-wrong}{%
\subsection{Who says that aristotle's view is
wrong?}\label{who-says-that-aristotles-view-is-wrong}}

\begin{quote}
\emph{summarized\_paragraph} : Aristotle believes that there are
processes in nature which are completed and regulated by a final state,
or end, towards which they tend. As Adams puts it, Aristotle has a much
stronger position on final causality than the Stoics. The Stoics believe
that there is no such thing as the end of a process, just the beginning
of a new process. \textbf{Adams says that Aristotle's view is wrong, and
that the process of growing a tree, for example, is not the end, but the
beginning.}
\end{quote}

\begin{quote}
\emph{avg\_grammar\_rating} : nan\\
\emph{avg\_answerability\_rating} : nan\\
\emph{sum\_yes\_meaningful} : 0\\
\emph{sum\_no\_meaning} : 0\\
\emph{sum\_maybe\_meaning} : 0
\end{quote}

\hypertarget{what-is-a-trees-process-of-doing}{%
\subsection{What is a tree's process of
doing?}\label{what-is-a-trees-process-of-doing}}

\begin{quote}
\emph{summarized\_paragraph} : Aristotle believes that there are
processes in nature which are completed and regulated by a final state,
or end, towards which they tend. As Adams puts it, Aristotle has a much
stronger position on final causality than the Stoics. The Stoics believe
that there is no such thing as the end of a process, just the beginning
of a new process. \textbf{Adams says that Aristotle's view is wrong, and
that the process of growing a tree, for example, is not the end, but the
beginning.}
\end{quote}

\begin{quote}
\emph{avg\_grammar\_rating} : nan\\
\emph{avg\_answerability\_rating} : nan\\
\emph{sum\_yes\_meaningful} : 0\\
\emph{sum\_no\_meaning} : 0\\
\emph{sum\_maybe\_meaning} : 0
\end{quote}

\hypertarget{what-must-be-the-causes-of-the-state-of-affairs}{%
\subsection{What must be the causes of the state of
affairs?}\label{what-must-be-the-causes-of-the-state-of-affairs}}

\begin{quote}
\emph{summarized\_paragraph} : There was a generally accepted
metatheory, namely that of Aristotle's Posterior Analytics, according to
which scientific demonstrations were syllogistic proofs. \textbf{In the
latter, the syllogisms involved must have middle terms that are causes
of the state of affairs which is to be demonstrated.} This gives a
theory of scientific reasoning in which the structure of the arguments
is intimately tied up with the structures of the causal chains that they
demonstrate. The second question is that of a meetingatheory.
\end{quote}

\begin{quote}
\emph{avg\_grammar\_rating} : nan\\
\emph{avg\_answerability\_rating} : nan\\
\emph{sum\_yes\_meaningful} : 0\\
\emph{sum\_no\_meaning} : 0\\
\emph{sum\_maybe\_meaning} : 0
\end{quote}

\hypertarget{the-middle-terms-in-a-syllogism-must-be-causes-of-what}{%
\subsection{The middle terms in a syllogism must be causes of
what?}\label{the-middle-terms-in-a-syllogism-must-be-causes-of-what}}

\begin{quote}
\emph{summarized\_paragraph} : There was a generally accepted
metatheory, namely that of Aristotle's Posterior Analytics, according to
which scientific demonstrations were syllogistic proofs. \textbf{In the
latter, the syllogisms involved must have middle terms that are causes
of the state of affairs which is to be demonstrated.} This gives a
theory of scientific reasoning in which the structure of the arguments
is intimately tied up with the structures of the causal chains that they
demonstrate. The second question is that of a meetingatheory.
\end{quote}

\begin{quote}
\emph{avg\_grammar\_rating} : nan\\
\emph{avg\_answerability\_rating} : nan\\
\emph{sum\_yes\_meaningful} : 0\\
\emph{sum\_no\_meaning} : 0\\
\emph{sum\_maybe\_meaning} : 0
\end{quote}

\hypertarget{aristotle-believes-that-processes-in-nature-are-completed-and-regulated-by-what}{%
\subsection{Aristotle believes that processes in nature are completed
and regulated by
what?}\label{aristotle-believes-that-processes-in-nature-are-completed-and-regulated-by-what}}

\begin{quote}
\emph{summarized\_paragraph} : \textbf{Aristotle believes that there are
processes in nature which are completed and regulated by a final state,
or end, towards which they tend.} As Adams puts it, Aristotle has a much
stronger position on final causality than the Stoics. The Stoics believe
that there is no such thing as the end of a process, just the beginning
of a new process. Adams says that Aristotle's view is wrong, and that
the process of growing a tree, for example, is not the end, but the
beginning.
\end{quote}

\begin{quote}
\emph{avg\_grammar\_rating} : nan\\
\emph{avg\_answerability\_rating} : nan\\
\emph{sum\_yes\_meaningful} : 0\\
\emph{sum\_no\_meaning} : 0\\
\emph{sum\_maybe\_meaning} : 0
\end{quote}

\hypertarget{aristotle-believes-that-processes-in-nature-are-what}{%
\subsection{Aristotle believes that processes in nature are
what?}\label{aristotle-believes-that-processes-in-nature-are-what}}

\begin{quote}
\emph{summarized\_paragraph} : \textbf{Aristotle believes that there are
processes in nature which are completed and regulated by a final state,
or end, towards which they tend.} As Adams puts it, Aristotle has a much
stronger position on final causality than the Stoics. The Stoics believe
that there is no such thing as the end of a process, just the beginning
of a new process. Adams says that Aristotle's view is wrong, and that
the process of growing a tree, for example, is not the end, but the
beginning.
\end{quote}

\begin{quote}
\emph{avg\_grammar\_rating} : nan\\
\emph{avg\_answerability\_rating} : nan\\
\emph{sum\_yes\_meaningful} : 0\\
\emph{sum\_no\_meaning} : 0\\
\emph{sum\_maybe\_meaning} : 0
\end{quote}

\hypertarget{the-importance-of-the-theory-of-evolution-led-to-careful-analysis-of-what}{%
\subsection{The importance of the theory of evolution led to careful
analysis of
what?}\label{the-importance-of-the-theory-of-evolution-led-to-careful-analysis-of-what}}

\begin{quote}
\emph{summarized\_paragraph} : There is a large body of medieval
literature which deals with general change of state. \textbf{It is
important because it led to a great deal of careful analysis of
continuous change.} This body of analysis was one of the contributing
currents which led to the invention of the calculus in the sixteenth
century. So, we can look at this literature from two viewpoints: either
from the tradition which it grew out of -- that of interpretations on
Aristotle's Physics and the picture of change found there -- or from the
viewpoint of what it lead to.
\end{quote}

\begin{quote}
\emph{avg\_grammar\_rating} : nan\\
\emph{avg\_answerability\_rating} : nan\\
\emph{sum\_yes\_meaningful} : 0\\
\emph{sum\_no\_meaning} : 0\\
\emph{sum\_maybe\_meaning} : 0
\end{quote}

\hypertarget{what-kind-of-causes-are-we-now-call}{%
\subsection{What kind of causes are we now
call?}\label{what-kind-of-causes-are-we-now-call}}

\begin{quote}
\emph{summarized\_paragraph} : Aristotle used the term `cause' in a
somewhat wider sense than is current nowadays. \textbf{Efficient causes
are what we would now simply call `causes' Final causes, however, are
problematic: a final cause is an end or a purpose, and, whereas it is
clear that rational agents act for the sake of ends, it is not clear
that much else does.} It seems clear to us that the causality of a
rationally pursued goal can be reduced to efficient causality.
\end{quote}

\begin{quote}
\emph{avg\_grammar\_rating} : nan\\
\emph{avg\_answerability\_rating} : nan\\
\emph{sum\_yes\_meaningful} : 0\\
\emph{sum\_no\_meaning} : 0\\
\emph{sum\_maybe\_meaning} : 0
\end{quote}

\hypertarget{did-avicenna-do-what-to-those-that-were-not-convinced-by}{%
\subsection{Did Avicenna do what to those that were not convinced
by?}\label{did-avicenna-do-what-to-those-that-were-not-convinced-by}}

\begin{quote}
\emph{summarized\_paragraph} : Ockham was very innovative, and altered
the landscape of medieval thought quite decisively. \textbf{He produced
a large number of interesting arguments, but, on the basis of these
arguments, he adopted positions which many of his contemporaries were
not convinced by.} So, although his arguments forced a reexamination of
many philosophical and theological doctrines, the thinkers influenced by
him took up a broad spectrum of positions, depending on which of his
doctrines they assented to and which of them they rejected. Walter
Chatton, for example, denied Ockham's position on intuitive cognition --
namely, that when we know singulars we do so by a separate intellectual
faculty distinct from sense perception.
\end{quote}

\begin{quote}
\emph{avg\_grammar\_rating} : nan\\
\emph{avg\_answerability\_rating} : nan\\
\emph{sum\_yes\_meaningful} : 0\\
\emph{sum\_no\_meaning} : 0\\
\emph{sum\_maybe\_meaning} : 0
\end{quote}

\hypertarget{what-type-of-arguments-did-avicenna-produce}{%
\subsection{What type of arguments did Avicenna
produce?}\label{what-type-of-arguments-did-avicenna-produce}}

\begin{quote}
\emph{summarized\_paragraph} : Ockham was very innovative, and altered
the landscape of medieval thought quite decisively. \textbf{He produced
a large number of interesting arguments, but, on the basis of these
arguments, he adopted positions which many of his contemporaries were
not convinced by.} So, although his arguments forced a reexamination of
many philosophical and theological doctrines, the thinkers influenced by
him took up a broad spectrum of positions, depending on which of his
doctrines they assented to and which of them they rejected. Walter
Chatton, for example, denied Ockham's position on intuitive cognition --
namely, that when we know singulars we do so by a separate intellectual
faculty distinct from sense perception.
\end{quote}

\begin{quote}
\emph{avg\_grammar\_rating} : nan\\
\emph{avg\_answerability\_rating} : nan\\
\emph{sum\_yes\_meaningful} : 0\\
\emph{sum\_no\_meaning} : 0\\
\emph{sum\_maybe\_meaning} : 0
\end{quote}

\hypertarget{what-case-can-be-examined-by-looking-at}{%
\subsection{What case can be examined by looking
at?}\label{what-case-can-be-examined-by-looking-at}}

\begin{quote}
\emph{summarized\_paragraph} : Ockham, like other fourteenth-century
theologians, frequently gives instances where we can make reliable
causal inferences. These arguments frequently rely on a theory of
natural kinds. For example, Ockham writes that we can come to know
causal propositions on the basis of experience. The answer to the first
question is quite straightforward. It is possible to make a reliable
causal inference from experience. \textbf{This can be done by looking at
the case of Buridan.} For more information on Buridan, visit the Buridan
website.
\end{quote}

\begin{quote}
\emph{avg\_grammar\_rating} : nan\\
\emph{avg\_answerability\_rating} : nan\\
\emph{sum\_yes\_meaningful} : 0\\
\emph{sum\_no\_meaning} : 0\\
\emph{sum\_maybe\_meaning} : 0
\end{quote}
