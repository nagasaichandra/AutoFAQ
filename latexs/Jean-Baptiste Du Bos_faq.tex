\hypertarget{jean-baptiste-du-bos}{%
\section{\texorpdfstring{\href{https://plato.stanford.edu/entries/du-bos/index.html}{Jean-Baptiste
Du Bos}}{Jean-Baptiste Du Bos}}\label{jean-baptiste-du-bos}}

\hypertarget{who-thought-that-art-was-valuable-for-its-own-sake}{%
\subsection{Who thought that art was valuable for its own
sake?}\label{who-thought-that-art-was-valuable-for-its-own-sake}}

\begin{quote}
\emph{summarized\_paragraph} : Some writers believe that artworks can be
valuable as a source of knowledge. At times Du Bos indicates that there
is something to this, and that art is valuable as more than the source
of valuable sentiments. \textbf{In the end, however, his considered
opinion is that, ``We can acquire some knowledge by reading a poem, but
this is scarcely the motive for opening the book'' Du Bos was aware that
some writers, including himself, believed that art was valuable for its
own sake.}
\end{quote}

\begin{quote}
\emph{avg\_grammar\_rating} : nan\\
\emph{avg\_answerability\_rating} : nan\\
\emph{sum\_yes\_meaningful} : 0\\
\emph{sum\_no\_meaning} : 0\\
\emph{sum\_maybe\_meaning} : 0
\end{quote}

\hypertarget{what-art-must-be-based-on-correct-and-rule---governed-reasoning}{%
\subsection{What art must be based on correct and rule - governed
reasoning?}\label{what-art-must-be-based-on-correct-and-rule---governed-reasoning}}

\begin{quote}
\emph{summarized\_paragraph} : According to Du Bos, we judge artworks by
means of our sentiments. He rejects the rationalist school of criticism
associated with writers such as Roland Fréart de Chambray. Du Bos mocks
these writers as `geometrical critics'. \textbf{In a similar vein, Bosse
wrote that the `noble art of painting must be based for the most part on
correct and rule-governed reasoning, which is to say, geometrical and,
consequently, demonstrative'}
\end{quote}

\begin{quote}
\emph{avg\_grammar\_rating} : nan\\
\emph{avg\_answerability\_rating} : nan\\
\emph{sum\_yes\_meaningful} : 0\\
\emph{sum\_no\_meaning} : 0\\
\emph{sum\_maybe\_meaning} : 0
\end{quote}

\hypertarget{what-must-painters-have-to-follow-in-order-to-paint}{%
\subsection{What must painters have to follow in order to
paint?}\label{what-must-painters-have-to-follow-in-order-to-paint}}

\begin{quote}
\emph{summarized\_paragraph} : Du Bos values what he calls
vraisemblance. \textbf{Painters, for example, must ``make a painting
consistent with what we know of the customs, habits, architecture, and
arms of the people that one intends to represent'' A work can, however,
be vra isemblable without being an imitation of the real world and
historical events.} Vra is not an end in itself as it was for some
earlier thinkers. The real goal of the arts is to arouse sentiments and
vraIsemblance is only a means of doing so.
\end{quote}

\begin{quote}
\emph{avg\_grammar\_rating} : nan\\
\emph{avg\_answerability\_rating} : nan\\
\emph{sum\_yes\_meaningful} : 0\\
\emph{sum\_no\_meaning} : 0\\
\emph{sum\_maybe\_meaning} : 0
\end{quote}

\hypertarget{what-kind-of-world-is-the-work-of-a-painter-not-an-imitation-of}{%
\subsection{What kind of world is the work of a painter not an imitation
of?}\label{what-kind-of-world-is-the-work-of-a-painter-not-an-imitation-of}}

\begin{quote}
\emph{summarized\_paragraph} : Du Bos values what he calls
vraisemblance. \textbf{Painters, for example, must ``make a painting
consistent with what we know of the customs, habits, architecture, and
arms of the people that one intends to represent'' A work can, however,
be vra isemblable without being an imitation of the real world and
historical events.} Vra is not an end in itself as it was for some
earlier thinkers. The real goal of the arts is to arouse sentiments and
vraIsemblance is only a means of doing so.
\end{quote}

\begin{quote}
\emph{avg\_grammar\_rating} : nan\\
\emph{avg\_answerability\_rating} : nan\\
\emph{sum\_yes\_meaningful} : 0\\
\emph{sum\_no\_meaning} : 0\\
\emph{sum\_maybe\_meaning} : 0
\end{quote}

\hypertarget{what-is-poetry-a-form-of}{%
\subsection{What is poetry a form of?}\label{what-is-poetry-a-form-of}}

\begin{quote}
\emph{summarized\_paragraph} : Du Bos asks which of the arts is best
able to arouse emotions. He believes that painting has an advantage over
poetry on the grounds that natural signs affect us more effectively. He
argues, however, that poetry when combined with music or acting (or
both, as in opera) has anadvantage over painting. \textbf{Du Bos: Poetry
is a form of art that should be used to express our emotions, rather
than as an art form in and of its own right.} For more information,
visit poetry.org.
\end{quote}

\begin{quote}
\emph{avg\_grammar\_rating} : nan\\
\emph{avg\_answerability\_rating} : nan\\
\emph{sum\_yes\_meaningful} : 0\\
\emph{sum\_no\_meaning} : 0\\
\emph{sum\_maybe\_meaning} : 0
\end{quote}

\hypertarget{what-is-poetry-a-form-of-1}{%
\subsection{What is poetry a form
of?}\label{what-is-poetry-a-form-of-1}}

\begin{quote}
\emph{summarized\_paragraph} : Du Bos asks which of the arts is best
able to arouse emotions. He believes that painting has an advantage over
poetry on the grounds that natural signs affect us more effectively. He
argues, however, that poetry when combined with music or acting (or
both, as in opera) has anadvantage over painting. \textbf{Du Bos: Poetry
is a form of art that should be used to express our emotions, rather
than as an art form in and of its own right.} For more information,
visit poetry.org.
\end{quote}

\begin{quote}
\emph{avg\_grammar\_rating} : nan\\
\emph{avg\_answerability\_rating} : nan\\
\emph{sum\_yes\_meaningful} : 0\\
\emph{sum\_no\_meaning} : 0\\
\emph{sum\_maybe\_meaning} : 0
\end{quote}

\hypertarget{what-type-of-stimulus-does-avicenna-say-a-soul-receives-when-it-receives-a-stimulus}{%
\subsection{What type of stimulus does Avicenna say a soul receives when
it receives a
stimulus?}\label{what-type-of-stimulus-does-avicenna-say-a-soul-receives-when-it-receives-a-stimulus}}

\begin{quote}
\emph{summarized\_paragraph} : The concept of sentiments is crucial to
Du Bos's thought. He seems to have been the writer who made talk of
sentiments so commonplace in eighteenth-century aesthetics. Du Bos never
precisely defines what he means when he speaks of sentiments. \textbf{In
what is, perhaps, the closest he comes to a definition he writes that,
``The first ideas born in the soul, when it receives a lively
stimulus,\ldots we call sentiments'' According toDu Bos, sentiments are
not produced by a special sort of aesthetic experience.} They are, as
already indicated, ordinary emotional responses.
\end{quote}

\begin{quote}
\emph{avg\_grammar\_rating} : nan\\
\emph{avg\_answerability\_rating} : nan\\
\emph{sum\_yes\_meaningful} : 0\\
\emph{sum\_no\_meaning} : 0\\
\emph{sum\_maybe\_meaning} : 0
\end{quote}

\hypertarget{according-to-du-bos-sentiments-are-ordinary-emotions-or-what}{%
\subsection{According to Du bos , sentiments are ordinary emotions or
what?}\label{according-to-du-bos-sentiments-are-ordinary-emotions-or-what}}

\begin{quote}
\emph{summarized\_paragraph} : Du Bos compares the sense of beauty to
gustatory taste and seems to have been among the first writers to do so.
\textbf{He writes that, ``We have in us a sense intended to judge the
value of works that imitate touching objects in nature'' and calls this
a ``sixth sense'' Despite believing that sentiments are ordinary
emotions or fainter copies of them, Du Bos posits a sense ofBeauty.} He
believes that we should judge works of art by their quality, rather than
their content.
\end{quote}

\begin{quote}
\emph{avg\_grammar\_rating} : nan\\
\emph{avg\_answerability\_rating} : nan\\
\emph{sum\_yes\_meaningful} : 0\\
\emph{sum\_no\_meaning} : 0\\
\emph{sum\_maybe\_meaning} : 0
\end{quote}

\hypertarget{what-kind-of-objects-does-du-bos-believe-we-imitate}{%
\subsection{What kind of objects does Du Bos believe we
imitate?}\label{what-kind-of-objects-does-du-bos-believe-we-imitate}}

\begin{quote}
\emph{summarized\_paragraph} : Du Bos compares the sense of beauty to
gustatory taste and seems to have been among the first writers to do so.
\textbf{He writes that, ``We have in us a sense intended to judge the
value of works that imitate touching objects in nature'' and calls this
a ``sixth sense'' Despite believing that sentiments are ordinary
emotions or fainter copies of them, Du Bos posits a sense ofBeauty.} He
believes that we should judge works of art by their quality, rather than
their content.
\end{quote}

\begin{quote}
\emph{avg\_grammar\_rating} : nan\\
\emph{avg\_answerability\_rating} : nan\\
\emph{sum\_yes\_meaningful} : 0\\
\emph{sum\_no\_meaning} : 0\\
\emph{sum\_maybe\_meaning} : 0
\end{quote}

\hypertarget{what-type-of-reflections-do-du-bos-give}{%
\subsection{What type of reflections do du bos
give?}\label{what-type-of-reflections-do-du-bos-give}}

\begin{quote}
\emph{summarized\_paragraph} : Du Bos says that the faithful depiction
of the passions suffices to make us afraid and make us resolve to avoid
them. This later passage is apparently influenced by Aristotle's Poetics
since Du Bos speaks of the purging of emotions. For example, we watch a
performance of Medea and we are horrified by the passion for vengeance
and resolve not to indulge in it. \textbf{Later in the Critical
Reflections Du Bos returns to the paradox of tragedy and gives a rather
different solution.}
\end{quote}

\begin{quote}
\emph{avg\_grammar\_rating} : nan\\
\emph{avg\_answerability\_rating} : nan\\
\emph{sum\_yes\_meaningful} : 0\\
\emph{sum\_no\_meaning} : 0\\
\emph{sum\_maybe\_meaning} : 0
\end{quote}

\hypertarget{what-does-he-do-in-the-second-instance}{%
\subsection{What does he do in the second
instance?}\label{what-does-he-do-in-the-second-instance}}

\begin{quote}
\emph{summarized\_paragraph} : Du Bos applied empiricism to the study of
art, as we have seen. According to Du Bos, judgements of works of art
must be empirical. In the first instance, Du Bos applied~empiricism~to
the study~of art. \textbf{He writes that ~``I~believe~that the best art
is the one that I can see.'' In the second instance, he writes that ``I
believe that the best work of art is that which I have seen in my own
eyes''.}
\end{quote}

\begin{quote}
\emph{avg\_grammar\_rating} : nan\\
\emph{avg\_answerability\_rating} : nan\\
\emph{sum\_yes\_meaningful} : 0\\
\emph{sum\_no\_meaning} : 0\\
\emph{sum\_maybe\_meaning} : 0
\end{quote}

\hypertarget{what-is-musical-imitation-similar-to-being-attracted-to-an-object}{%
\subsection{What is musical imitation similar to being attracted to an
object?}\label{what-is-musical-imitation-similar-to-being-attracted-to-an-object}}

\begin{quote}
\emph{summarized\_paragraph} : Du Bos believes that the experience of
musical imitations arouses emotion. Music arouses emotions in the same
way that poetry and painting do. The emotion aroused will be the emotion
that would be aroused by the object represented. Unlike some modern
advocates of the resemblance theory, Du
Bos~believes~that~musical~imitations~are~emotional~in~the~same~way~as~poetry~and~painting~are.
\textbf{The~experience~of musical~imitation~is~similar~to
that~of~being~attracted~to an object.}
\end{quote}

\begin{quote}
\emph{avg\_grammar\_rating} : nan\\
\emph{avg\_answerability\_rating} : nan\\
\emph{sum\_yes\_meaningful} : 0\\
\emph{sum\_no\_meaning} : 0\\
\emph{sum\_maybe\_meaning} : 0
\end{quote}

\hypertarget{what-is-the-experience-of-being-attracted-to-an-object}{%
\subsection{What is the experience of being attracted to an
object?}\label{what-is-the-experience-of-being-attracted-to-an-object}}

\begin{quote}
\emph{summarized\_paragraph} : Du Bos believes that the experience of
musical imitations arouses emotion. Music arouses emotions in the same
way that poetry and painting do. The emotion aroused will be the emotion
that would be aroused by the object represented. Unlike some modern
advocates of the resemblance theory, Du
Bos~believes~that~musical~imitations~are~emotional~in~the~same~way~as~poetry~and~painting~are.
\textbf{The~experience~of musical~imitation~is~similar~to
that~of~being~attracted~to an object.}
\end{quote}

\begin{quote}
\emph{avg\_grammar\_rating} : nan\\
\emph{avg\_answerability\_rating} : nan\\
\emph{sum\_yes\_meaningful} : 0\\
\emph{sum\_no\_meaning} : 0\\
\emph{sum\_maybe\_meaning} : 0
\end{quote}

\hypertarget{what-did-everyone-do-with-a-work-of-art}{%
\subsection{What did everyone do with a work of
art?}\label{what-did-everyone-do-with-a-work-of-art}}

\begin{quote}
\emph{summarized\_paragraph} : \textbf{The problem, as Hume clearly
recognized, was that not everyone receives the same sentiments from a
work of art.} This leads to the conclusion that a work has value for
some people and not for others. It seems that there is a fact that
Milton is a greater poet than Ogilby, even if some people prefer
Ogilsby. But thorough-going subjectivism is counterintuitive, Hume also
recognized, as well as the fact that some people like Ogilsbrough more
than others.
\end{quote}

\begin{quote}
\emph{avg\_grammar\_rating} : nan\\
\emph{avg\_answerability\_rating} : nan\\
\emph{sum\_yes\_meaningful} : 0\\
\emph{sum\_no\_meaning} : 0\\
\emph{sum\_maybe\_meaning} : 0
\end{quote}

\hypertarget{what-did-voltaire-think-of-du-bos}{%
\subsection{What did Voltaire think of du
bos?}\label{what-did-voltaire-think-of-du-bos}}

\begin{quote}
\emph{summarized\_paragraph} : The influence of Du Bos on aesthetics in
France began with his impact on Voltaire. \textbf{He described Du Bos as
``a very wise, very learned, and very esteemed man' According to
Voltaire, ``All artists read with profit his Reflections on poetry,
painting, and music' Voltaire agreed with Du Bos that there have been
four ages where the arts particularly flourished.} He also adopted one
ofDu Bos's controversial views: the belief that national character is
affected by matters such as climate.
\end{quote}

\begin{quote}
\emph{avg\_grammar\_rating} : nan\\
\emph{avg\_answerability\_rating} : nan\\
\emph{sum\_yes\_meaningful} : 0\\
\emph{sum\_no\_meaning} : 0\\
\emph{sum\_maybe\_meaning} : 0
\end{quote}

\hypertarget{what-did-voltaire-call-du-bos}{%
\subsection{What did Voltaire call du
bos?}\label{what-did-voltaire-call-du-bos}}

\begin{quote}
\emph{summarized\_paragraph} : The influence of Du Bos on aesthetics in
France began with his impact on Voltaire. \textbf{He described Du Bos as
``a very wise, very learned, and very esteemed man' According to
Voltaire, ``All artists read with profit his Reflections on poetry,
painting, and music' Voltaire agreed with Du Bos that there have been
four ages where the arts particularly flourished.} He also adopted one
ofDu Bos's controversial views: the belief that national character is
affected by matters such as climate.
\end{quote}

\begin{quote}
\emph{avg\_grammar\_rating} : nan\\
\emph{avg\_answerability\_rating} : nan\\
\emph{sum\_yes\_meaningful} : 0\\
\emph{sum\_no\_meaning} : 0\\
\emph{sum\_maybe\_meaning} : 0
\end{quote}

\hypertarget{what-does-this-conclusion-lead-to}{%
\subsection{What does this conclusion lead
to?}\label{what-does-this-conclusion-lead-to}}

\begin{quote}
\emph{summarized\_paragraph} : The problem, as Hume clearly recognized,
was that not everyone receives the same sentiments from a work of art.
\textbf{This leads to the conclusion that a work has value for some
people and not for others.} It seems that there is a fact that Milton is
a greater poet than Ogilby, even if some people prefer Ogilsby. But
thorough-going subjectivism is counterintuitive, Hume also recognized,
as well as the fact that some people like Ogilsbrough more than others.
\end{quote}

\begin{quote}
\emph{avg\_grammar\_rating} : nan\\
\emph{avg\_answerability\_rating} : nan\\
\emph{sum\_yes\_meaningful} : 0\\
\emph{sum\_no\_meaning} : 0\\
\emph{sum\_maybe\_meaning} : 0
\end{quote}

\hypertarget{what-did-avicenna-say-were-made-to-please-us-by-touching-us}{%
\subsection{What did Avicenna say were made to please us by touching
us?}\label{what-did-avicenna-say-were-made-to-please-us-by-touching-us}}

\begin{quote}
\emph{summarized\_paragraph} : He writes that the evaluation of art ``is
not left to reason. It must submit to the judgement that sentiment
pronounces. Sentiment is the competent judge of this issue.'' He goes on
to compare the. evaluation of an artwork to making a judgement about a
ragout. Rather, ``We taste the ragout and\ldots we know that it is good.
\textbf{It is the same with works of wit and pictures made to please us
by touching us'' He writes that ``reason is of no use here'}
\end{quote}

\begin{quote}
\emph{avg\_grammar\_rating} : nan\\
\emph{avg\_answerability\_rating} : nan\\
\emph{sum\_yes\_meaningful} : 0\\
\emph{sum\_no\_meaning} : 0\\
\emph{sum\_maybe\_meaning} : 0
\end{quote}

\hypertarget{was-batteux-influenced-by-du-bos}{%
\subsection{Was batteux influenced by du
bos?}\label{was-batteux-influenced-by-du-bos}}

\begin{quote}
\emph{summarized\_paragraph} : \textbf{Batteux is known to have read the
Critical Reflections and was apparently influenced by Du Bos.} Du Bos
seems to have been one of the important sources of the resemblance
theory of musical expression in the eighteenth century. He adds that
``music is half-formed in the words that express some emotion. It takes
only a little art to turn the words into music'' . These passages from
The Fine Arts are little more than paraphrases of passages from the
Criticalreflections.
\end{quote}

\begin{quote}
\emph{avg\_grammar\_rating} : nan\\
\emph{avg\_answerability\_rating} : nan\\
\emph{sum\_yes\_meaningful} : 0\\
\emph{sum\_no\_meaning} : 0\\
\emph{sum\_maybe\_meaning} : 0
\end{quote}

\hypertarget{who-was-known-to-have-read-critical-reflections}{%
\subsection{Who was known to have read critical
reflections?}\label{who-was-known-to-have-read-critical-reflections}}

\begin{quote}
\emph{summarized\_paragraph} : \textbf{Batteux is known to have read the
Critical Reflections and was apparently influenced by Du Bos.} Du Bos
seems to have been one of the important sources of the resemblance
theory of musical expression in the eighteenth century. He adds that
``music is half-formed in the words that express some emotion. It takes
only a little art to turn the words into music'' . These passages from
The Fine Arts are little more than paraphrases of passages from the
Criticalreflections.
\end{quote}

\begin{quote}
\emph{avg\_grammar\_rating} : nan\\
\emph{avg\_answerability\_rating} : nan\\
\emph{sum\_yes\_meaningful} : 0\\
\emph{sum\_no\_meaning} : 0\\
\emph{sum\_maybe\_meaning} : 0
\end{quote}

\hypertarget{what-does-du-bos-indicate-that-art-is}{%
\subsection{What does du bos indicate that art
is?}\label{what-does-du-bos-indicate-that-art-is}}

\begin{quote}
\emph{summarized\_paragraph} : Some writers believe that artworks can be
valuable as a source of knowledge. \textbf{At times Du Bos indicates
that there is something to this, and that art is valuable as more than
the source of valuable sentiments.} In the end, however, his considered
opinion is that, ``We can acquire some knowledge by reading a poem, but
this is scarcely the motive for opening the book'' Du Bos was aware that
some writers, including himself, believed that art was valuable for its
own sake.
\end{quote}

\begin{quote}
\emph{avg\_grammar\_rating} : nan\\
\emph{avg\_answerability\_rating} : nan\\
\emph{sum\_yes\_meaningful} : 0\\
\emph{sum\_no\_meaning} : 0\\
\emph{sum\_maybe\_meaning} : 0
\end{quote}

\hypertarget{what-did-harveys-about-the-circulation-of-blood}{%
\subsection{What did Harvey's about the circulation of
blood?}\label{what-did-harveys-about-the-circulation-of-blood}}

\begin{quote}
\emph{summarized\_paragraph} : Du Bos makes clear that the circulation
of blood could only have been discovered a posteriori. People have
accepted Harvey's hypothesis, Du Bos writes, because it is the only way
to explain blood circulation. In another passage, he discusses Harvey''s
discovery of the discovery of blood circulation and makes clear it could
only be found by a~preliminary~investigation. \textbf{In another
section, he talks about Harvey``s discovery~of the circulation~of blood
and how it was discovered.}
\end{quote}

\begin{quote}
\emph{avg\_grammar\_rating} : nan\\
\emph{avg\_answerability\_rating} : nan\\
\emph{sum\_yes\_meaningful} : 0\\
\emph{sum\_no\_meaning} : 0\\
\emph{sum\_maybe\_meaning} : 0
\end{quote}

\hypertarget{what-did-harveys-about-the-circulation-of-blood-1}{%
\subsection{What did Harvey's about the circulation of
blood?}\label{what-did-harveys-about-the-circulation-of-blood-1}}

\begin{quote}
\emph{summarized\_paragraph} : Du Bos makes clear that the circulation
of blood could only have been discovered a posteriori. People have
accepted Harvey's hypothesis, Du Bos writes, because it is the only way
to explain blood circulation. In another passage, he discusses Harvey''s
discovery of the discovery of blood circulation and makes clear it could
only be found by a~preliminary~investigation. \textbf{In another
section, he talks about Harvey``s discovery~of the circulation~of blood
and how it was discovered.}
\end{quote}

\begin{quote}
\emph{avg\_grammar\_rating} : nan\\
\emph{avg\_answerability\_rating} : nan\\
\emph{sum\_yes\_meaningful} : 0\\
\emph{sum\_no\_meaning} : 0\\
\emph{sum\_maybe\_meaning} : 0
\end{quote}

\hypertarget{what-was-this-passage-written-in}{%
\subsection{What was this passage written
in?}\label{what-was-this-passage-written-in}}

\begin{quote}
\emph{summarized\_paragraph} : This passage is clearly an endorsement of
empiricism and a rejection of Cartesian rationalism. \textbf{This
passage was written in response to a question about the nature of
reality in the world.} The author was responding to an earlier version
of this article that suggested that the world was not rational. He said
that the question was whether the world is rational or not, and the
answer was that it is not. He then went on to say that this was not the
case, but that he was talking about the world as he knew it, not
Cartesian rationality.
\end{quote}

\begin{quote}
\emph{avg\_grammar\_rating} : nan\\
\emph{avg\_answerability\_rating} : nan\\
\emph{sum\_yes\_meaningful} : 0\\
\emph{sum\_no\_meaning} : 0\\
\emph{sum\_maybe\_meaning} : 0
\end{quote}

\hypertarget{according-to-popper-what-quality-of-the-general-public-was-the-least-corrupted-by-prejudice}{%
\subsection{According to Popper , what quality of the general public was
the least corrupted by
prejudice?}\label{according-to-popper-what-quality-of-the-general-public-was-the-least-corrupted-by-prejudice}}

\begin{quote}
\emph{summarized\_paragraph} : Du Bos was unaware of the full extent of
the problem caused by subjectivism. Likely this was because he believed
that everyone has very much the same sentiments. \textbf{In fact, he
believed the general public was the best judge of the value of artworks
because their taste was least corrupted by prejudice.} He believed that
the public was best judge because they were the least prejudiced. He
also believed that they had the best taste in art because they did not
have to be prejudiced to judge it.
\end{quote}

\begin{quote}
\emph{avg\_grammar\_rating} : nan\\
\emph{avg\_answerability\_rating} : nan\\
\emph{sum\_yes\_meaningful} : 0\\
\emph{sum\_no\_meaning} : 0\\
\emph{sum\_maybe\_meaning} : 0
\end{quote}
