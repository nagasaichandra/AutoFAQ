\hypertarget{logicism-and-neologicism}{%
\section{\texorpdfstring{\href{https://plato.stanford.edu/entries/logicism/index.html}{Logicism
and
Neologicism}}{Logicism and Neologicism}}\label{logicism-and-neologicism}}

\hypertarget{what-is-short-and-easy}{%
\subsection{What is short and easy?}\label{what-is-short-and-easy}}

\begin{quote}
\emph{summarized\_paragraph} : Wright sketched a derivation of the
Dedekind-Peano axioms from Hume's Principle. The deductions would be
carried out in standard second-order logic. Such a system is unfree with
respect to its number-abstractive terms. \textbf{This point holds even
if the second- order logic in question is a free logic in the official
sense of not being committed to the theorem-scheme for any well-formed
singular term t. The proof of this point is short and easy, and is like
the one given in §1.2.}
\end{quote}

\begin{quote}
\emph{avg\_grammar\_rating} : nan\\
\emph{avg\_answerability\_rating} : nan\\
\emph{sum\_yes\_meaningful} : 0\\
\emph{sum\_no\_meaning} : 0\\
\emph{sum\_maybe\_meaning} : 0
\end{quote}

\hypertarget{whose-three-primitives-are-0-natural-number-and-successor}{%
\subsection{Whose three primitives are ` 0 ' , ` natural number ' and `
successor'?}\label{whose-three-primitives-are-0-natural-number-and-successor}}

\begin{quote}
\emph{summarized\_paragraph} : \textbf{Frege's procedure is in the form
of defining Peano's three primitives `0', `natural number' and
`successor' It is not necessary to use any axioms of set existence
except in introducing terms of the form `NxFx' And in proving , so that
the argument could be carried out by taking as an axiom.} \ldots{} We
can put {[}Frege'S procedure{]} in the form of~defining~Peano`s~three
primitives, and proving Peano's~axioms.
\end{quote}

\begin{quote}
\emph{avg\_grammar\_rating} : nan\\
\emph{avg\_answerability\_rating} : nan\\
\emph{sum\_yes\_meaningful} : 0\\
\emph{sum\_no\_meaning} : 0\\
\emph{sum\_maybe\_meaning} : 0
\end{quote}

\hypertarget{how-is-the-same-similar-to-the-same}{%
\subsection{How is the same similar to the
same?}\label{how-is-the-same-similar-to-the-same}}

\begin{quote}
\emph{summarized\_paragraph} : Russell sought to preserve Frege's
approach to defining cardinal numbers as classes of similar-sized
classes. Russell, however, sought to retain the idea that cardinal
numbers can be divided into similar-size classes. \textbf{He wrote that
the cardinal numbers are `similar' to each other in the same way that
they are ``similar'' to one another in terms of size.`Similar' is the
same as `the same' in the sense of being `in the same place'.}
\end{quote}

\begin{quote}
\emph{avg\_grammar\_rating} : nan\\
\emph{avg\_answerability\_rating} : nan\\
\emph{sum\_yes\_meaningful} : 0\\
\emph{sum\_no\_meaning} : 0\\
\emph{sum\_maybe\_meaning} : 0
\end{quote}

\hypertarget{what-were-the-intended-recipients-of-the-sentence}{%
\subsection{What were the intended recipients of the
sentence?}\label{what-were-the-intended-recipients-of-the-sentence}}

\begin{quote}
\emph{summarized\_paragraph} : The Grundgesetze were intended to supply
some instruction in this regard. \textbf{And the footnote to
``Umfang's'' ends with the sentence ``I assume that it is known what the
extension of a concept is.'' For those who nevertheless needed some
instruction, they were meant to supply it.} For more information, visit
the website of the German Academy of Sciences and Humanities at:
http://www.scholarshop.org/grundgesettingze.
\end{quote}

\begin{quote}
\emph{avg\_grammar\_rating} : nan\\
\emph{avg\_answerability\_rating} : nan\\
\emph{sum\_yes\_meaningful} : 0\\
\emph{sum\_no\_meaning} : 0\\
\emph{sum\_maybe\_meaning} : 0
\end{quote}

\hypertarget{what-is-a-term-of}{%
\subsection{What is a term of?}\label{what-is-a-term-of}}

\begin{quote}
\emph{summarized\_paragraph} : Wright expressed reservations over
whether ``x = x'' counts as a sortal predicate eligible to be prefixed
by ``the number of x such that'' Now Wright was inquiring after what
``is wanted for the exorcism of anti-zero'' . \textbf{His considered
answer is that a term of the form }\#xFx will denote a number only if
the concept F is both sortal and not indefinitely extensible. So Wright
subsequently hoped to make actual what he had previously alleged to be
impossible to imagine.
\end{quote}

\begin{quote}
\emph{avg\_grammar\_rating} : nan\\
\emph{avg\_answerability\_rating} : nan\\
\emph{sum\_yes\_meaningful} : 0\\
\emph{sum\_no\_meaning} : 0\\
\emph{sum\_maybe\_meaning} : 0
\end{quote}

\hypertarget{what-is-the-answer-to-the-question-of-the-form}{%
\subsection{What is the answer to the question of the
form?}\label{what-is-the-answer-to-the-question-of-the-form}}

\begin{quote}
\emph{summarized\_paragraph} : Wright expressed reservations over
whether ``x = x'' counts as a sortal predicate eligible to be prefixed
by ``the number of x such that'' Now Wright was inquiring after what
``is wanted for the exorcism of anti-zero'' . \textbf{His considered
answer is that a term of the form }\#xFx will denote a number only if
the concept F is both sortal and not indefinitely extensible. So Wright
subsequently hoped to make actual what he had previously alleged to be
impossible to imagine.
\end{quote}

\begin{quote}
\emph{avg\_grammar\_rating} : nan\\
\emph{avg\_answerability\_rating} : nan\\
\emph{sum\_yes\_meaningful} : 0\\
\emph{sum\_no\_meaning} : 0\\
\emph{sum\_maybe\_meaning} : 0
\end{quote}

\hypertarget{what-should-one-be-in-to-count-any-finite-collection-of-objects}{%
\subsection{What should one be in to count any finite collection of
objects?}\label{what-should-one-be-in-to-count-any-finite-collection-of-objects}}

\begin{quote}
\emph{summarized\_paragraph} : Boolos provided an ingenious model (which
had been anticipated informally by Geach ) to allay the misgiving about
the consistency of full second-order logic with HP. Note, however, that
this consistency proof works only when FA is taken on its own. FA should
be applicable not only to concrete objects, but also to abstract
mathematical entities such as real numbers and sets. \textbf{Provided
only that one has a criterion of identity for the objects in question,
one should be in a position to count any finite collection of them.}
\end{quote}

\begin{quote}
\emph{avg\_grammar\_rating} : nan\\
\emph{avg\_answerability\_rating} : nan\\
\emph{sum\_yes\_meaningful} : 0\\
\emph{sum\_no\_meaning} : 0\\
\emph{sum\_maybe\_meaning} : 0
\end{quote}

\hypertarget{what-type-of-expression-is-required-in-a-language-that-is-rich-enough-to-provide-the-two-forms-of-expression-in-what}{%
\subsection{What type of expression is required in a language that is
rich enough to provide the two forms of expression in
what?}\label{what-type-of-expression-is-required-in-a-language-that-is-rich-enough-to-provide-the-two-forms-of-expression-in-what}}

\begin{quote}
\emph{summarized\_paragraph} : \textbf{Frege: If we countenance these
two different ways of `carving' one and the same propositional content,
then we shall require, in whatever language is rich enough to provide
the two forms of expression in question, the following logical
equivalence, indicated by the two-way deducibility sign ⊣⊢.} If, with
Frege, we countenant these two ways of `carving' one and. the same
content, we shall need, in. whatever language, to provide, in whichever
language, an equivalence between the two. forms in question.
\end{quote}

\begin{quote}
\emph{avg\_grammar\_rating} : nan\\
\emph{avg\_answerability\_rating} : nan\\
\emph{sum\_yes\_meaningful} : 0\\
\emph{sum\_no\_meaning} : 0\\
\emph{sum\_maybe\_meaning} : 0
\end{quote}

\hypertarget{what-is-not-committed-to-the-concept-of-numbers-as-an-object}{%
\subsection{What is not committed to the concept of numbers as an
object?}\label{what-is-not-committed-to-the-concept-of-numbers-as-an-object}}

\begin{quote}
\emph{summarized\_paragraph} : The form on the right, within a language
devoid of the operator \#, is completely innocent of any commitment to
numbers as objects. If such a language is extended, however, by adding
\# to its stock of logical expressions, then one is thereby able to
express the form onThe left, which is number-committal. The form on The
Right, within the same language, is not number-Committal, but
number-free. \textbf{This is because the language is not committed to
the concept of numbers as an object.}
\end{quote}

\begin{quote}
\emph{avg\_grammar\_rating} : nan\\
\emph{avg\_answerability\_rating} : nan\\
\emph{sum\_yes\_meaningful} : 0\\
\emph{sum\_no\_meaning} : 0\\
\emph{sum\_maybe\_meaning} : 0
\end{quote}

\hypertarget{what-is-not-committed-to-the-concept-of-numbers-as-an-object-1}{%
\subsection{What is not committed to the concept of numbers as an
object?}\label{what-is-not-committed-to-the-concept-of-numbers-as-an-object-1}}

\begin{quote}
\emph{summarized\_paragraph} : The form on the right, within a language
devoid of the operator \#, is completely innocent of any commitment to
numbers as objects. If such a language is extended, however, by adding
\# to its stock of logical expressions, then one is thereby able to
express the form onThe left, which is number-committal. The form on The
Right, within the same language, is not number-Committal, but
number-free. \textbf{This is because the language is not committed to
the concept of numbers as an object.}
\end{quote}

\begin{quote}
\emph{avg\_grammar\_rating} : nan\\
\emph{avg\_answerability\_rating} : nan\\
\emph{sum\_yes\_meaningful} : 0\\
\emph{sum\_no\_meaning} : 0\\
\emph{sum\_maybe\_meaning} : 0
\end{quote}

\hypertarget{the-word-r---bars-can-be-used-to-refer-to-any-of-what}{%
\subsection{The word r - bars can be used to refer to any of
what?}\label{the-word-r---bars-can-be-used-to-refer-to-any-of-what}}

\begin{quote}
\emph{summarized\_paragraph} : The second ancillary notion we shall
express here as ``x is R-barred by F'', or ``F R-bars x', is defined
thus: ``X is F R- bars x'' and it is defined as `x is F-bars X'.
\textbf{This is the definition of the ``R-bars'' part of the word `R' in
the `F' part of `X' and it can be used to refer to any of a number of
things.}
\end{quote}

\begin{quote}
\emph{avg\_grammar\_rating} : nan\\
\emph{avg\_answerability\_rating} : nan\\
\emph{sum\_yes\_meaningful} : 0\\
\emph{sum\_no\_meaning} : 0\\
\emph{sum\_maybe\_meaning} : 0
\end{quote}

\hypertarget{what-is-another-term-for-the-universe}{%
\subsection{What is another term for the
universe?}\label{what-is-another-term-for-the-universe}}

\begin{quote}
\emph{summarized\_paragraph} : There is an obvious corollary to the idea
that the universe is made up of atoms and molecules. This is the case in
the universe of atoms, molecules, and atoms. In the world of atoms there
are also molecules, or atoms, or molecules. The idea of the universe
being made up by atoms is that there are atoms, and molecules, that make
up the universe. In this case, there are molecules, which in turn make
up atoms. \textbf{This leads to the concept of the Universe, or the
universe, as we know it.}
\end{quote}

\begin{quote}
\emph{avg\_grammar\_rating} : nan\\
\emph{avg\_answerability\_rating} : nan\\
\emph{sum\_yes\_meaningful} : 0\\
\emph{sum\_no\_meaning} : 0\\
\emph{sum\_maybe\_meaning} : 0
\end{quote}

\hypertarget{what-does-n-mean}{%
\subsection{What does " n " mean?}\label{what-does-n-mean}}

\begin{quote}
\emph{summarized\_paragraph} : Zalta defines an equinumerosity relation
among properties with respect to ordinary objects. \textbf{With ≈ in
hand, Zalta offers the notion of a number :~~~``N'' is the number of
objects that can be considered concrete.} The number of concrete objects
can be as large as~``n''~or~as~small as ``n'' is~the number~of
objects~that~can be considered~concrete~by the~equinumerity~relations.
\end{quote}

\begin{quote}
\emph{avg\_grammar\_rating} : nan\\
\emph{avg\_answerability\_rating} : nan\\
\emph{sum\_yes\_meaningful} : 0\\
\emph{sum\_no\_meaning} : 0\\
\emph{sum\_maybe\_meaning} : 0
\end{quote}

\hypertarget{the-number-of-what-can-be-shown-to-be-at-least-one-or-more-in-the-form-of-a-function}{%
\subsection{The number of what can be shown to be at least one or more
in the form of a
function?}\label{the-number-of-what-can-be-shown-to-be-at-least-one-or-more-in-the-form-of-a-function}}

\begin{quote}
\emph{summarized\_paragraph} : The recursion axioms are provable using
the recursionAxioms. The recursion Axioms can be used to prove a number
of different types of recursion. \textbf{The number of recursions can be
shown to be at least one or more in the form of a function.} The result
is a number which is provable by recursion and the recursions are
recursion-based. The results can be proved by using
the~recursion~axioms, which are~properly~proved~using recursion~and
the~reconcurrence~axis.
\end{quote}

\begin{quote}
\emph{avg\_grammar\_rating} : nan\\
\emph{avg\_answerability\_rating} : nan\\
\emph{sum\_yes\_meaningful} : 0\\
\emph{sum\_no\_meaning} : 0\\
\emph{sum\_maybe\_meaning} : 0
\end{quote}

\hypertarget{the-number-of-recursions-can-be-shown-to-be-at-least-one-or-more-in-the-form-of-what}{%
\subsection{The number of recursions can be shown to be at least one or
more in the form of
what?}\label{the-number-of-recursions-can-be-shown-to-be-at-least-one-or-more-in-the-form-of-what}}

\begin{quote}
\emph{summarized\_paragraph} : The recursion axioms are provable using
the recursionAxioms. The recursion Axioms can be used to prove a number
of different types of recursion. \textbf{The number of recursions can be
shown to be at least one or more in the form of a function.} The result
is a number which is provable by recursion and the recursions are
recursion-based. The results can be proved by using
the~recursion~axioms, which are~properly~proved~using recursion~and
the~reconcurrence~axis.
\end{quote}

\begin{quote}
\emph{avg\_grammar\_rating} : nan\\
\emph{avg\_answerability\_rating} : nan\\
\emph{sum\_yes\_meaningful} : 0\\
\emph{sum\_no\_meaning} : 0\\
\emph{sum\_maybe\_meaning} : 0
\end{quote}

\hypertarget{what-kind-of-point-can-we-make-about-the-use-of-number---abstraction-in-mathematics}{%
\subsection{What kind of point can we make about the use of number -
abstraction in
mathematics?}\label{what-kind-of-point-can-we-make-about-the-use-of-number---abstraction-in-mathematics}}

\begin{quote}
\emph{summarized\_paragraph} : Frege's point here can be made
watertight, and general, after providing a few words of explanation
about notation for number-abstraction. We shall show how Frege's point
here could be made more general, and watertight. \textbf{We shall also
show how this point could be used to make a more general point about the
use of number- Abstraction in mathematics.} We will conclude with an
explanation of how to use number-Abstraction to make our point more
general.
\end{quote}

\begin{quote}
\emph{avg\_grammar\_rating} : nan\\
\emph{avg\_answerability\_rating} : nan\\
\emph{sum\_yes\_meaningful} : 0\\
\emph{sum\_no\_meaning} : 0\\
\emph{sum\_maybe\_meaning} : 0
\end{quote}

\hypertarget{what-could-be-used-to-make-a-more-general-point-about-the-use-of-number---abstraction-in-mathematics}{%
\subsection{What could be used to make a more general point about the
use of number - abstraction in
mathematics?}\label{what-could-be-used-to-make-a-more-general-point-about-the-use-of-number---abstraction-in-mathematics}}

\begin{quote}
\emph{summarized\_paragraph} : Frege's point here can be made
watertight, and general, after providing a few words of explanation
about notation for number-abstraction. We shall show how Frege's point
here could be made more general, and watertight. \textbf{We shall also
show how this point could be used to make a more general point about the
use of number- Abstraction in mathematics.} We will conclude with an
explanation of how to use number-Abstraction to make our point more
general.
\end{quote}

\begin{quote}
\emph{avg\_grammar\_rating} : nan\\
\emph{avg\_answerability\_rating} : nan\\
\emph{sum\_yes\_meaningful} : 0\\
\emph{sum\_no\_meaning} : 0\\
\emph{sum\_maybe\_meaning} : 0
\end{quote}

\hypertarget{what-is-the-first-thing-that-one-makes-to-justify-the-statement-that-n-is-the-same-as-2}{%
\subsection{What is the first thing that one makes to justify the
statement that n is the same as
2?}\label{what-is-the-first-thing-that-one-makes-to-justify-the-statement-that-n-is-the-same-as-2}}

\begin{quote}
\emph{summarized\_paragraph} : Numerosity can be used as a justifying
ground for the subsequent statement that the number of Fs is identical
to 2. Of course (returning to our example where n = 2), one can consider
matters in the converse logical direction. \textbf{If one makes the
numerosity assertion first, then one can regard that as a justification
for the next statement that n is the same as 2.} If n is greater than 2,
then n is equal to 2, and if n is less than 2 then it is not.
\end{quote}

\begin{quote}
\emph{avg\_grammar\_rating} : nan\\
\emph{avg\_answerability\_rating} : nan\\
\emph{sum\_yes\_meaningful} : 0\\
\emph{sum\_no\_meaning} : 0\\
\emph{sum\_maybe\_meaning} : 0
\end{quote}

\hypertarget{what-assertion-is-used-to-justify-a-statement-that-n-is-the-same-as-2}{%
\subsection{What assertion is used to justify a statement that n is the
same as
2?}\label{what-assertion-is-used-to-justify-a-statement-that-n-is-the-same-as-2}}

\begin{quote}
\emph{summarized\_paragraph} : Numerosity can be used as a justifying
ground for the subsequent statement that the number of Fs is identical
to 2. Of course (returning to our example where n = 2), one can consider
matters in the converse logical direction. \textbf{If one makes the
numerosity assertion first, then one can regard that as a justification
for the next statement that n is the same as 2.} If n is greater than 2,
then n is equal to 2, and if n is less than 2 then it is not.
\end{quote}

\begin{quote}
\emph{avg\_grammar\_rating} : nan\\
\emph{avg\_answerability\_rating} : nan\\
\emph{sum\_yes\_meaningful} : 0\\
\emph{sum\_no\_meaning} : 0\\
\emph{sum\_maybe\_meaning} : 0
\end{quote}

\hypertarget{what-is-similar-to-the-same-number-of-other-classes-in-the-same-class}{%
\subsection{What is similar to the same number of other classes in the
same
class?}\label{what-is-similar-to-the-same-number-of-other-classes-in-the-same-class}}

\begin{quote}
\emph{summarized\_paragraph} : The cardinal number of a class α is
defined as the class of all classes similar to α. Two classes are
similar when there is a one-one relation between them. The cardinal
number is the number of classes that are similar to each other.
\textbf{It is the same as the number that is similar to the same number
of other classes in the same class.} For more information on the
cardinal number, see: cardinal number~of~classes~and~the~class~number~of
classes in a class.
\end{quote}

\begin{quote}
\emph{avg\_grammar\_rating} : nan\\
\emph{avg\_answerability\_rating} : nan\\
\emph{sum\_yes\_meaningful} : 0\\
\emph{sum\_no\_meaning} : 0\\
\emph{sum\_maybe\_meaning} : 0
\end{quote}

\hypertarget{what-can-be-used-as-a-justificationing-ground-for-the-statement-that-the-number-of-fs-is-identical-to-2}{%
\subsection{What can be used as a justificationing ground for the
statement that the number of fs is identical to
2?}\label{what-can-be-used-as-a-justificationing-ground-for-the-statement-that-the-number-of-fs-is-identical-to-2}}

\begin{quote}
\emph{summarized\_paragraph} : \textbf{Numerosity can be used as a
justifying ground for the subsequent statement that the number of Fs is
identical to 2.} Of course (returning to our example where n = 2), one
can consider matters in the converse logical direction. If one makes the
numerosity assertion first, then one can regard that as a justification
for the next statement that n is the same as 2. If n is greater than 2,
then n is equal to 2, and if n is less than 2 then it is not.
\end{quote}

\begin{quote}
\emph{avg\_grammar\_rating} : nan\\
\emph{avg\_answerability\_rating} : nan\\
\emph{sum\_yes\_meaningful} : 0\\
\emph{sum\_no\_meaning} : 0\\
\emph{sum\_maybe\_meaning} : 0
\end{quote}

\hypertarget{what-is-numerosity-used-for-in-the-statement-that-the-number-of-fs-is-identical-to-2}{%
\subsection{What is numerosity used for in the statement that the number
of fs is identical to
2?}\label{what-is-numerosity-used-for-in-the-statement-that-the-number-of-fs-is-identical-to-2}}

\begin{quote}
\emph{summarized\_paragraph} : \textbf{Numerosity can be used as a
justifying ground for the subsequent statement that the number of Fs is
identical to 2.} Of course (returning to our example where n = 2), one
can consider matters in the converse logical direction. If one makes the
numerosity assertion first, then one can regard that as a justification
for the next statement that n is the same as 2. If n is greater than 2,
then n is equal to 2, and if n is less than 2 then it is not.
\end{quote}

\begin{quote}
\emph{avg\_grammar\_rating} : nan\\
\emph{avg\_answerability\_rating} : nan\\
\emph{sum\_yes\_meaningful} : 0\\
\emph{sum\_no\_meaning} : 0\\
\emph{sum\_maybe\_meaning} : 0
\end{quote}

\hypertarget{what-is-the-same-as-the-previous-one}{%
\subsection{What is the same as the previous
one?}\label{what-is-the-same-as-the-previous-one}}

\begin{quote}
\emph{summarized\_paragraph} : We have (suppressing parentheses) we
have. Hence (suppresses parentheses) We have. (Suppressing parentheses,
we have). We have, therefore, (suppressed parentheses, and so on) Hence
we have, and this is the result we seek. We have the result. We now
have. The result is. The result. is the conclusion. The conclusion.
\textbf{is that the result is the same as the previous one, and we have
the same result.}
\end{quote}

\begin{quote}
\emph{avg\_grammar\_rating} : nan\\
\emph{avg\_answerability\_rating} : nan\\
\emph{sum\_yes\_meaningful} : 0\\
\emph{sum\_no\_meaning} : 0\\
\emph{sum\_maybe\_meaning} : 0
\end{quote}

\hypertarget{what-tells-us-that-the-numbers-are}{%
\subsection{What tells us that the numbers
are?}\label{what-tells-us-that-the-numbers-are}}

\begin{quote}
\emph{summarized\_paragraph} : ** HP tells us that the numbers **\#xFx
and \#xGx will be identical if and only if the predicate-extensions that
they respectively number are in one-one correspondence (under some
two-place relation R) Another way of expressing this latter condition is
to say that F and G are equinumerous. HP says that the two numbers are
identical if they are in a correspondence under a two-places relation R.
This is not always the case, however, and sometimes they are not
identical at all.
\end{quote}

\begin{quote}
\emph{avg\_grammar\_rating} : nan\\
\emph{avg\_answerability\_rating} : nan\\
\emph{sum\_yes\_meaningful} : 0\\
\emph{sum\_no\_meaning} : 0\\
\emph{sum\_maybe\_meaning} : 0
\end{quote}

\hypertarget{what-is-another-term-for-class}{%
\subsection{What is another term for
class?}\label{what-is-another-term-for-class}}

\begin{quote}
\emph{summarized\_paragraph} : \textbf{With impredicative `class
abstracts', the existence of such a class could not be guaranteed as a
matter of logic.} Russell had to postulate that such classes existed.
This came to be regarded as detracting from their status as would-be
logical objects, and revealing them instead as no more than mathematical
posits. Their existence was once again a synthetic a priori matter,
rather than one of analytic necessity and certainty. They were once
again considered to be mathematical~posits~rather than logical objects.
\end{quote}

\begin{quote}
\emph{avg\_grammar\_rating} : nan\\
\emph{avg\_answerability\_rating} : nan\\
\emph{sum\_yes\_meaningful} : 0\\
\emph{sum\_no\_meaning} : 0\\
\emph{sum\_maybe\_meaning} : 0
\end{quote}
