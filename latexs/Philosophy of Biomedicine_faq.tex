\hypertarget{philosophy-of-biomedicine}{%
\section{\texorpdfstring{\href{https://plato.stanford.edu/entries/biomedicine/index.html}{Philosophy
of
Biomedicine}}{Philosophy of Biomedicine}}\label{philosophy-of-biomedicine}}

\hypertarget{what-does-not-duplicate-in-the-entry-on-concepts-of-health-and-disease}{%
\subsection{What does not duplicate in the entry on concepts of health
and
disease?}\label{what-does-not-duplicate-in-the-entry-on-concepts-of-health-and-disease}}

\begin{quote}
\emph{summarized\_paragraph} : The preceding section on purpose in
biomedicine leads directly to the related issue of how health and
disease are conceptualized. \textbf{This is something that will only be
dealt with very briefly here, in part to avoid duplicating the content
in the entry on Concepts of Health and Disease.} Debates over the
meaning of health and Disease are central to philosophy of medicine.
This entry will not attempt to summarize that complex literature, and
rather focus on how conceptions ofhealth and disease relate to
biomedics.
\end{quote}

\begin{quote}
\emph{avg\_grammar\_rating} : nan\\
\emph{avg\_answerability\_rating} : nan\\
\emph{sum\_yes\_meaningful} : 0\\
\emph{sum\_no\_meaning} : 0\\
\emph{sum\_maybe\_meaning} : 0
\end{quote}

\hypertarget{what-is-a-major-source-of-data-banking}{%
\subsection{What is a major source of data
banking?}\label{what-is-a-major-source-of-data-banking}}

\begin{quote}
\emph{summarized\_paragraph} : The US National Cancer Institute defines
biomedicine as synonymous with ``allopathic medicine, conventional
medicine, mainstream medicine, orthodox medicine, and Western medicine''
By contrast, A. E. Clarke, Mamo, Fishman, Shim, and Fosket takes an
expansive view of the nature of biomediine. \textbf{They argue that it
is an evolving entity, a cohesive and developing whole that consists of
elements ranging from the assertion that good health is a personal moral
obligation to the increasing reliance on ``computerization and data
banking''}
\end{quote}

\begin{quote}
\emph{avg\_grammar\_rating} : nan\\
\emph{avg\_answerability\_rating} : nan\\
\emph{sum\_yes\_meaningful} : 0\\
\emph{sum\_no\_meaning} : 0\\
\emph{sum\_maybe\_meaning} : 0
\end{quote}

\hypertarget{what-is-the-hard-problem-of-consciousness-like-to-make-sense-of-the-patient}{%
\subsection{What is the hard problem of consciousness like to make sense
of the
patient?}\label{what-is-the-hard-problem-of-consciousness-like-to-make-sense-of-the-patient}}

\begin{quote}
\emph{summarized\_paragraph} : The human mind and consciousness attract
a special sense of awe. \textbf{While the hard problem of consciousness
is not all that different from the problems facing attempts to make
sense of the patient as a whole.} Patients live; their bodies function
every second of the day, via an astounding series of interconnected
processes. It stretches the imagination to think that a vital spirit or
such is giving unity to each life. It is another way to think we humans
are simply skin bags of chemical reactions. We are not the same thing.
\end{quote}

\begin{quote}
\emph{avg\_grammar\_rating} : nan\\
\emph{avg\_answerability\_rating} : nan\\
\emph{sum\_yes\_meaningful} : 0\\
\emph{sum\_no\_meaning} : 0\\
\emph{sum\_maybe\_meaning} : 0
\end{quote}

\hypertarget{biochemical-puzzles-are-not-yet-what}{%
\subsection{Biochemical puzzles are not yet
what?}\label{biochemical-puzzles-are-not-yet-what}}

\begin{quote}
\emph{summarized\_paragraph} : \textbf{The potential for success and
record of prior successes is an argument for at least operating under
the assumption that biomedical scholars ought to continue treating
biomedical mysteries as biochemical puzzles for which the relevant
pieces have not yet been identified or assembled.} In other words, the
potential for successful research is a strong argument for treating them
as a biochemical puzzle, rather than as a mystery, as is the record of
successful research in this area in the U.S. and elsewhere. In the past,
this has led some to call them ``biological mysteries''
\end{quote}

\begin{quote}
\emph{avg\_grammar\_rating} : nan\\
\emph{avg\_answerability\_rating} : nan\\
\emph{sum\_yes\_meaningful} : 0\\
\emph{sum\_no\_meaning} : 0\\
\emph{sum\_maybe\_meaning} : 0
\end{quote}

\hypertarget{whose-values-and-motivations-are-often-used-in-these-critiques}{%
\subsection{Whose values and motivations are often used in these
critiques?}\label{whose-values-and-motivations-are-often-used-in-these-critiques}}

\begin{quote}
\emph{summarized\_paragraph} : Both EBM and other branches of
biomedicine are united by a valuing of precise measurement. While
precise measurement is achievable in the biochemistry lab, the practical
and philosophical challenges are thornier when doing measurement at the
level of the whole person or the population. \textbf{A common theme
shared by these critiques is a concern about the fickleness and
contingency of measurement leaving much room for practitioners' values
and motivations to shape the results of the measurement process, for
good or for ill.}
\end{quote}

\begin{quote}
\emph{avg\_grammar\_rating} : nan\\
\emph{avg\_answerability\_rating} : nan\\
\emph{sum\_yes\_meaningful} : 0\\
\emph{sum\_no\_meaning} : 0\\
\emph{sum\_maybe\_meaning} : 0
\end{quote}

\hypertarget{what-part-of-a-system-is-malfunctioning}{%
\subsection{What part of a system is
malfunctioning?}\label{what-part-of-a-system-is-malfunctioning}}

\begin{quote}
\emph{summarized\_paragraph} : Disease is reduced to the state of a
system in which there are malfunctioning parts (pathologies in body
parts) Biomedicine takes illness seriously, but the philosophical
framework of biomedicines leaves no space for the notions of existential
transformation as part of illness. See also the discussion of
phenomenology in Section 3.4 of this article. The article was originally
published in the online edition of The New York Review of Books, volume
2, issue 1, issue 2.
\end{quote}

\begin{quote}
\emph{avg\_grammar\_rating} : nan\\
\emph{avg\_answerability\_rating} : nan\\
\emph{sum\_yes\_meaningful} : 0\\
\emph{sum\_no\_meaning} : 0\\
\emph{sum\_maybe\_meaning} : 0
\end{quote}

\hypertarget{what-type-of-transformation-does-biomedicine-leave-no-space-for}{%
\subsection{What type of transformation does biomedicine leave no space
for?}\label{what-type-of-transformation-does-biomedicine-leave-no-space-for}}

\begin{quote}
\emph{summarized\_paragraph} : Disease is reduced to the state of a
system in which there are malfunctioning parts (pathologies in body
parts) Biomedicine takes illness seriously, but the philosophical
framework of biomedicines leaves no space for the notions of existential
transformation as part of illness. See also the discussion of
phenomenology in Section 3.4 of this article. The article was originally
published in the online edition of The New York Review of Books, volume
2, issue 1, issue 2.
\end{quote}

\begin{quote}
\emph{avg\_grammar\_rating} : nan\\
\emph{avg\_answerability\_rating} : nan\\
\emph{sum\_yes\_meaningful} : 0\\
\emph{sum\_no\_meaning} : 0\\
\emph{sum\_maybe\_meaning} : 0
\end{quote}

\hypertarget{what-type-of-phenomena-does-theorize-the-domain-of-disease-is-restricted-to}{%
\subsection{What type of phenomena does theorize the domain of disease
is restricted
to?}\label{what-type-of-phenomena-does-theorize-the-domain-of-disease-is-restricted-to}}

\begin{quote}
\emph{summarized\_paragraph} : Broadbent: Biomedicine actively excludes
consideration of entities and processes that don't fit into its
worldview. The strongest case for this chauvinism rests more on
pragmatic grounds than on philosophical assumptions, he says.
\textbf{Broadbent: The assumption is that ``the domain of disease and
its causes is restricted to solely biological, chemical, and physical
phenomena'' This is a powerful argument, though it cuts both ways, he
writes.} Section 5 discusses critiques attempting to undercut the value
of what biomedicina has indeed built while operating under that
approach.
\end{quote}

\begin{quote}
\emph{avg\_grammar\_rating} : nan\\
\emph{avg\_answerability\_rating} : nan\\
\emph{sum\_yes\_meaningful} : 0\\
\emph{sum\_no\_meaning} : 0\\
\emph{sum\_maybe\_meaning} : 0
\end{quote}

\hypertarget{the-potential-for-successful-research-is-a-strong-argument-for-what}{%
\subsection{The potential for successful research is a strong argument
for
what?}\label{the-potential-for-successful-research-is-a-strong-argument-for-what}}

\begin{quote}
\emph{summarized\_paragraph} : The potential for success and record of
prior successes is an argument for at least operating under the
assumption that biomedical scholars ought to continue treating
biomedical mysteries as biochemical puzzles for which the relevant
pieces have not yet been identified or assembled. \textbf{In other
words, the potential for successful research is a strong argument for
treating them as a biochemical puzzle, rather than as a mystery, as is
the record of successful research in this area in the U.S. and
elsewhere.} In the past, this has led some to call them ``biological
mysteries''
\end{quote}

\begin{quote}
\emph{avg\_grammar\_rating} : nan\\
\emph{avg\_answerability\_rating} : nan\\
\emph{sum\_yes\_meaningful} : 0\\
\emph{sum\_no\_meaning} : 0\\
\emph{sum\_maybe\_meaning} : 0
\end{quote}

\hypertarget{what-type-of-research-is-a-strong-argument-for-treating-them-as-a-mystery}{%
\subsection{What type of research is a strong argument for treating them
as a
mystery?}\label{what-type-of-research-is-a-strong-argument-for-treating-them-as-a-mystery}}

\begin{quote}
\emph{summarized\_paragraph} : The potential for success and record of
prior successes is an argument for at least operating under the
assumption that biomedical scholars ought to continue treating
biomedical mysteries as biochemical puzzles for which the relevant
pieces have not yet been identified or assembled. \textbf{In other
words, the potential for successful research is a strong argument for
treating them as a biochemical puzzle, rather than as a mystery, as is
the record of successful research in this area in the U.S. and
elsewhere.} In the past, this has led some to call them ``biological
mysteries''
\end{quote}

\begin{quote}
\emph{avg\_grammar\_rating} : nan\\
\emph{avg\_answerability\_rating} : nan\\
\emph{sum\_yes\_meaningful} : 0\\
\emph{sum\_no\_meaning} : 0\\
\emph{sum\_maybe\_meaning} : 0
\end{quote}

\hypertarget{what-is-the-birth-of-the-clinic-written-as}{%
\subsection{What is the birth of the clinic written
as?}\label{what-is-the-birth-of-the-clinic-written-as}}

\begin{quote}
\emph{summarized\_paragraph} : Michel Foucault's work, and use of the
concepts biopower and biopolitics, remain touchstones for much of the
critical discourse surrounding biomedicine. His critique of modern
medicine is part of a career critiquing other aspects of modernity,
including the related topic of psychiatry. \textbf{His famous work The
Birth of the Clinic is written as a history, though in the process it
highlighted aspects of biomedics that other scholars went on to critique
as well.}
\end{quote}

\begin{quote}
\emph{avg\_grammar\_rating} : nan\\
\emph{avg\_answerability\_rating} : nan\\
\emph{sum\_yes\_meaningful} : 0\\
\emph{sum\_no\_meaning} : 0\\
\emph{sum\_maybe\_meaning} : 0
\end{quote}

\hypertarget{are-clinical-trials-highly-susceptible-to-manipulations-to-the-experimental-setup}{%
\subsection{Are clinical trials highly susceptible to manipulations to
the experimental
setup?}\label{are-clinical-trials-highly-susceptible-to-manipulations-to-the-experimental-setup}}

\begin{quote}
\emph{summarized\_paragraph} : A naturalistic concept of disease does
not prevent social processes from altering the standards and practices
of how the boundaries of these natural categories are drawn in practice.
Chronic diseases such as cardiovascular diseases and type 2 diabetes
have been targeted by drug companies to not only create new treatments
but to redraw the boundaries between healthy vs.~pathological.
\textbf{These efforts are in part accomplished via the design of
clinical trials, which have the dangerous distinction of having very
high epistemic value in the biomedical community, while remaining highly
susceptible to manipulations to the experimental setup.}
\end{quote}

\begin{quote}
\emph{avg\_grammar\_rating} : nan\\
\emph{avg\_answerability\_rating} : nan\\
\emph{sum\_yes\_meaningful} : 0\\
\emph{sum\_no\_meaning} : 0\\
\emph{sum\_maybe\_meaning} : 0
\end{quote}

\hypertarget{do-clinical-trials-have-high-or-low-epistemic-value-in-the-biomedical-community}{%
\subsection{Do clinical trials have high or low epistemic value in the
biomedical
community?}\label{do-clinical-trials-have-high-or-low-epistemic-value-in-the-biomedical-community}}

\begin{quote}
\emph{summarized\_paragraph} : A naturalistic concept of disease does
not prevent social processes from altering the standards and practices
of how the boundaries of these natural categories are drawn in practice.
Chronic diseases such as cardiovascular diseases and type 2 diabetes
have been targeted by drug companies to not only create new treatments
but to redraw the boundaries between healthy vs.~pathological.
\textbf{These efforts are in part accomplished via the design of
clinical trials, which have the dangerous distinction of having very
high epistemic value in the biomedical community, while remaining highly
susceptible to manipulations to the experimental setup.}
\end{quote}

\begin{quote}
\emph{avg\_grammar\_rating} : nan\\
\emph{avg\_answerability\_rating} : nan\\
\emph{sum\_yes\_meaningful} : 0\\
\emph{sum\_no\_meaning} : 0\\
\emph{sum\_maybe\_meaning} : 0
\end{quote}

\hypertarget{what-does-a-change-in-the-body-and-capacities-of-an-ill-persons-illness-affect}{%
\subsection{What does a change in the body and capacities of an ill
person's illness
affect?}\label{what-does-a-change-in-the-body-and-capacities-of-an-ill-persons-illness-affect}}

\begin{quote}
\emph{summarized\_paragraph} : philosophical analysis is needed to fully
appreciate the existential transformation illness brings about. This
transformation cannot be accounted for as merely physical or mental
dysfunction. \textbf{Rather, there is a need for a view of personhood as
embodied, situated, and enactive, in order to explain how local changes
to the ill person's body and capacities modify her existence globally.}
\ldots we must enlist philosophical analysis in order for us to fully
understand the existential transformations that illness can bring
about.''
\end{quote}

\begin{quote}
\emph{avg\_grammar\_rating} : nan\\
\emph{avg\_answerability\_rating} : nan\\
\emph{sum\_yes\_meaningful} : 0\\
\emph{sum\_no\_meaning} : 0\\
\emph{sum\_maybe\_meaning} : 0
\end{quote}

\hypertarget{what-is-known-for-being-dehumanizing}{%
\subsection{What is known for being
dehumanizing?}\label{what-is-known-for-being-dehumanizing}}

\begin{quote}
\emph{summarized\_paragraph} : Narrative medicine offers a revision to
the biomedical model alternative based around the centrality of the
story or narrative in human life. This notion that narrative is central
to human experience offers a variety of potential operationalizations in
clinical biomedical practice. It places additional value on the patient
consultation and asks for improved active listening skills among
clinicians. \textbf{Such listening has many potential benefits,
including the potential to understand the meaning of what human dignity
means for a given patient, an important benefit given that biomedical
settings are known for being ``dehumanizing''}
\end{quote}

\begin{quote}
\emph{avg\_grammar\_rating} : nan\\
\emph{avg\_answerability\_rating} : nan\\
\emph{sum\_yes\_meaningful} : 0\\
\emph{sum\_no\_meaning} : 0\\
\emph{sum\_maybe\_meaning} : 0
\end{quote}

\hypertarget{what-is-known-for-being-dehumanizing-1}{%
\subsection{What is known for being
dehumanizing?}\label{what-is-known-for-being-dehumanizing-1}}

\begin{quote}
\emph{summarized\_paragraph} : Narrative medicine offers a revision to
the biomedical model alternative based around the centrality of the
story or narrative in human life. This notion that narrative is central
to human experience offers a variety of potential operationalizations in
clinical biomedical practice. It places additional value on the patient
consultation and asks for improved active listening skills among
clinicians. \textbf{Such listening has many potential benefits,
including the potential to understand the meaning of what human dignity
means for a given patient, an important benefit given that biomedical
settings are known for being ``dehumanizing''}
\end{quote}

\begin{quote}
\emph{avg\_grammar\_rating} : nan\\
\emph{avg\_answerability\_rating} : nan\\
\emph{sum\_yes\_meaningful} : 0\\
\emph{sum\_no\_meaning} : 0\\
\emph{sum\_maybe\_meaning} : 0
\end{quote}

\hypertarget{what-is-one-of-the-fundamental-features-of-a-biomedical-perspective}{%
\subsection{What is one of the fundamental features of a biomedical
perspective?}\label{what-is-one-of-the-fundamental-features-of-a-biomedical-perspective}}

\begin{quote}
\emph{summarized\_paragraph} : \textbf{Among the many features of a
biomedical perspective, three stand out as fundamental regarding its
approach to investigating disease.} They are:. The role of the human
body in the study of disease, as well as its role in the discovery of
new treatments and treatments, and the role of our knowledge of the body
in these discoveries. The importance of our ability to learn from our
bodies, and to learn how to use them to treat disease, is a key feature
of the biomedical perspective. It is also a key part of our
understanding of the nature of disease.
\end{quote}

\begin{quote}
\emph{avg\_grammar\_rating} : nan\\
\emph{avg\_answerability\_rating} : nan\\
\emph{sum\_yes\_meaningful} : 0\\
\emph{sum\_no\_meaning} : 0\\
\emph{sum\_maybe\_meaning} : 0
\end{quote}

\hypertarget{what-is-one-of-the-fundamental-features-of-a-biomedical-perspective-1}{%
\subsection{What is one of the fundamental features of a biomedical
perspective?}\label{what-is-one-of-the-fundamental-features-of-a-biomedical-perspective-1}}

\begin{quote}
\emph{summarized\_paragraph} : \textbf{Among the many features of a
biomedical perspective, three stand out as fundamental regarding its
approach to investigating disease.} They are:. The role of the human
body in the study of disease, as well as its role in the discovery of
new treatments and treatments, and the role of our knowledge of the body
in these discoveries. The importance of our ability to learn from our
bodies, and to learn how to use them to treat disease, is a key feature
of the biomedical perspective. It is also a key part of our
understanding of the nature of disease.
\end{quote}

\begin{quote}
\emph{avg\_grammar\_rating} : nan\\
\emph{avg\_answerability\_rating} : nan\\
\emph{sum\_yes\_meaningful} : 0\\
\emph{sum\_no\_meaning} : 0\\
\emph{sum\_maybe\_meaning} : 0
\end{quote}

\hypertarget{what-can-processes-and-events-be-viewed-in}{%
\subsection{What can processes and events be viewed
in?}\label{what-can-processes-and-events-be-viewed-in}}

\begin{quote}
\emph{summarized\_paragraph} : A process ontology considers causal
relationships between events and processes to be foundational, rather
than things. It implies complex causal relationships which may be said
to be multi-factorial and non-linear. A process Ontology may be
described as a ``process-based ontology'' or a process-centered
ontology. It may also be called a process theory or a processes-based
theory. \textbf{It is based on the idea that processes and events are
one and the same, and that they can be viewed in different ways.}
\end{quote}

\begin{quote}
\emph{avg\_grammar\_rating} : nan\\
\emph{avg\_answerability\_rating} : nan\\
\emph{sum\_yes\_meaningful} : 0\\
\emph{sum\_no\_meaning} : 0\\
\emph{sum\_maybe\_meaning} : 0
\end{quote}

\hypertarget{what-is-used-to-determine-the-existence-of-the-senses}{%
\subsection{What is used to determine the existence of the
senses?}\label{what-is-used-to-determine-the-existence-of-the-senses}}

\begin{quote}
\emph{summarized\_paragraph} : The use of instrumentation is real and
exists, according to the metaphysical thesis that only what is
ascertainable by means of the five senses and by extension, the use of
instruments is real. The use of Instrumentation is Real and exists is
real according to metaphysical thesis. The metaphysical thesis is that
only. what is~ ascertainable~by means of~the five senses~is real
and~existence. By extension, by extension the use. of instrumentations
is real by extension by extension. \textbf{the metaphysical~s thesis
that~only what is ascertained~by the~five senses and by extension~the
use of~instrumentation is~real and exists.}
\end{quote}

\begin{quote}
\emph{avg\_grammar\_rating} : nan\\
\emph{avg\_answerability\_rating} : nan\\
\emph{sum\_yes\_meaningful} : 0\\
\emph{sum\_no\_meaning} : 0\\
\emph{sum\_maybe\_meaning} : 0
\end{quote}

\hypertarget{in-what-part-of-the-world-did-biomedicine-form}{%
\subsection{In what part of the world did biomedicine
form?}\label{in-what-part-of-the-world-did-biomedicine-form}}

\begin{quote}
\emph{summarized\_paragraph} : \textbf{Insofar as biomedicine formed in
the West, it is helpful to see it in contrast to the medical and
philosophical traditions that preceded it.} The most influential Western
tradition prior to biomededicine is the Hippocratic tradition. At the
core of Hippocratic medicine were two views. First, it made a commitment
to methodological naturalism. Second, it understood health as a matter
of balance---and disease as amatter of imbalance---of the humors. in
turn directly tied to a much larger cosmology of the elements (e.g.,
blood is linked to air, the springtime, a combination of heat and
moisture, the heart)
\end{quote}

\begin{quote}
\emph{avg\_grammar\_rating} : nan\\
\emph{avg\_answerability\_rating} : nan\\
\emph{sum\_yes\_meaningful} : 0\\
\emph{sum\_no\_meaning} : 0\\
\emph{sum\_maybe\_meaning} : 0
\end{quote}

\hypertarget{what-may-or-may-not-qualify-as-a-paradigm}{%
\subsection{What may or may not qualify as a
paradigm?}\label{what-may-or-may-not-qualify-as-a-paradigm}}

\begin{quote}
\emph{summarized\_paragraph} : **Biomedicine may or may not qualify as
one of Thomas Kuhn's ``paradigms' Biomedicines may also qualify as a
different way of conceiving of the way that research communities
intellectually cohere.** It''s a question of how biomedicin serves as a
means of organizing and guiding research. The terms ``framework'' and
``approach'' are used here to describe biomededicine, in an attempt to
avoid narrowly overcommitting to any particular philosophical system.
\end{quote}

\begin{quote}
\emph{avg\_grammar\_rating} : nan\\
\emph{avg\_answerability\_rating} : nan\\
\emph{sum\_yes\_meaningful} : 0\\
\emph{sum\_no\_meaning} : 0\\
\emph{sum\_maybe\_meaning} : 0
\end{quote}

\hypertarget{what-does-the-biological-chauvinism-side-of-the-coin-of-biological-chauvinism-do-to-the-biomedical-framework}{%
\subsection{What does the biological chauvinism side of the coin of
biological chauvinism do to the biomedical
framework?}\label{what-does-the-biological-chauvinism-side-of-the-coin-of-biological-chauvinism-do-to-the-biomedical-framework}}

\begin{quote}
\emph{summarized\_paragraph} : Some philosophers of biomedicine have
attempted to make headway in this question. They ask what it means for
biomedical sciences to be `chauvinist' and whether this is proper.
\textbf{Curiously, there are two sides to the coin of biological
chauvinism: one that directly supports the biomedical framework and one
that challenges it.} Both help shed light on the relationship between
biology and medicine and help to define what makes patients and their
bodies special. For more information, visit the Biomedical Chauvinism
website.
\end{quote}

\begin{quote}
\emph{avg\_grammar\_rating} : nan\\
\emph{avg\_answerability\_rating} : nan\\
\emph{sum\_yes\_meaningful} : 0\\
\emph{sum\_no\_meaning} : 0\\
\emph{sum\_maybe\_meaning} : 0
\end{quote}

\hypertarget{what-does-the-biological-chauvinism-side-of-the-coin-of-biological-chauvinism-do-to-the-biomedical-framework-1}{%
\subsection{What does the biological chauvinism side of the coin of
biological chauvinism do to the biomedical
framework?}\label{what-does-the-biological-chauvinism-side-of-the-coin-of-biological-chauvinism-do-to-the-biomedical-framework-1}}

\begin{quote}
\emph{summarized\_paragraph} : Some philosophers of biomedicine have
attempted to make headway in this question. They ask what it means for
biomedical sciences to be `chauvinist' and whether this is proper.
\textbf{Curiously, there are two sides to the coin of biological
chauvinism: one that directly supports the biomedical framework and one
that challenges it.} Both help shed light on the relationship between
biology and medicine and help to define what makes patients and their
bodies special. For more information, visit the Biomedical Chauvinism
website.
\end{quote}

\begin{quote}
\emph{avg\_grammar\_rating} : nan\\
\emph{avg\_answerability\_rating} : nan\\
\emph{sum\_yes\_meaningful} : 0\\
\emph{sum\_no\_meaning} : 0\\
\emph{sum\_maybe\_meaning} : 0
\end{quote}

\hypertarget{what-is-the-use-of-phenomenology-to-do-to-describe-human-lived-experience}{%
\subsection{What is the use of phenomenology to do to describe human
lived
experience?}\label{what-is-the-use-of-phenomenology-to-do-to-describe-human-lived-experience}}

\begin{quote}
\emph{summarized\_paragraph} : ``Biochauvinism'' is the view that there
is something philosophically unique about biological organisms. Wolfe
finds that sort of view inside the influential phenomenological work of
Maurice Merleau-Ponty. The view that life has some animating entity
(along the lines of spirit) that animates matter into a living being is
one form of biochauvinistism. \textbf{Another form is the use of
phenomenology to assert that human lived experience partly operates
within a space-time context that is different from that of a rock.}
\end{quote}

\begin{quote}
\emph{avg\_grammar\_rating} : nan\\
\emph{avg\_answerability\_rating} : nan\\
\emph{sum\_yes\_meaningful} : 0\\
\emph{sum\_no\_meaning} : 0\\
\emph{sum\_maybe\_meaning} : 0
\end{quote}
