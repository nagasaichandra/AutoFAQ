\hypertarget{cusanus-nicolaus-nicolas-of-cusa}{%
\section{\texorpdfstring{\href{https://plato.stanford.edu/entries/cusanus/index.html}{Cusanus,
Nicolaus {[}Nicolas of
Cusa{]}}}{Cusanus, Nicolaus {[}Nicolas of Cusa{]}}}\label{cusanus-nicolaus-nicolas-of-cusa}}

\hypertarget{how-is-the-presence-of-god-and-gods-identity-with-things-not-to-be-thought-of-as}{%
\subsection{How is the presence of god and god's identity with things
not to be thought of
as?}\label{how-is-the-presence-of-god-and-gods-identity-with-things-not-to-be-thought-of-as}}

\begin{quote}
\emph{summarized\_paragraph} : Christian Neoplatonism is distinctive for
its ability to hold together dialectically in thought the insight it
provides about this asymmetrical, non-reciprocal connection between God
and creatures. God penetrates and surpasses or exceeds each thing God
creates and encompasses. Creatures are thus themselves real with the
limited sort of independence they manifest, yet they are at once in God
and indeed one with God without being themselves divine. \textbf{This
means that the presence of God and God's identity with things is not to
be thought as the kind of reciprocity, say, that two created physical
things have.}
\end{quote}

\begin{quote}
\emph{avg\_grammar\_rating} : 5.0\\
\emph{avg\_answerability\_rating} : 5.0\\
\emph{sum\_yes\_meaningful} : 3\\
\emph{sum\_no\_meaning} : 0\\
\emph{sum\_maybe\_meaning} : 0
\end{quote}

\hypertarget{nicholas-said-that-assimilation-in-perception-is-a-matter-of-reasons-active-selecting-and-managing-the-deliverances-of-what}{%
\subsection{Nicholas said that assimilation in perception is a matter of
reason's active selecting and managing the deliverances of
what?}\label{nicholas-said-that-assimilation-in-perception-is-a-matter-of-reasons-active-selecting-and-managing-the-deliverances-of-what}}

\begin{quote}
\emph{summarized\_paragraph} : Nicholas asks us to imagine wax informed
by mind in the way mind informs our capacities for sensing. \textbf{He
proposes that mind so imagined could ``form the wax to every shape
presented to it.'' Nicholas is saying, in effect, that ``assimilation''
in perception is indeed a matter of reason's active selecting and
managing the deliverances of sense and imagination that result from our
encounters with perceptible things.} We are not mere passive recipients
of colors, sounds, textures and so on, he writes.
\end{quote}

\begin{quote}
\emph{avg\_grammar\_rating} : 3.8\\
\emph{avg\_answerability\_rating} : 5.0\\
\emph{sum\_yes\_meaningful} : 3\\
\emph{sum\_no\_meaning} : 1\\
\emph{sum\_maybe\_meaning} : 0
\end{quote}

\hypertarget{what-is-the-wall-of-paradise}{%
\subsection{What is the wall of
paradise?}\label{what-is-the-wall-of-paradise}}

\begin{quote}
\emph{summarized\_paragraph} : Cusanus leads us through a series of
reflections on seeing and on the face of God. God is located beyond both
imaginative exercise and conceptual understanding. \textbf{Nicholas
symbolizes our approach to this beyond by encouraging us to enter ``into
a certain secret and hidden silence wherein there is no knowledge or
concept of a face,'' characterizing it as an ``obscuring mist, haze,
darkness or ignorance.'' He invokes the coincidence of opposites from On
Learned Ignorance and proposes his second central metaphor: the wall of
paradise.}
\end{quote}

\begin{quote}
\emph{avg\_grammar\_rating} : 5.0\\
\emph{avg\_answerability\_rating} : 4.3\\
\emph{sum\_yes\_meaningful} : 3\\
\emph{sum\_no\_meaning} : 0\\
\emph{sum\_maybe\_meaning} : 0
\end{quote}

\hypertarget{what-symbolizes-our-approach-to-this-beyond}{%
\subsection{What symbolizes our approach to this
beyond?}\label{what-symbolizes-our-approach-to-this-beyond}}

\begin{quote}
\emph{summarized\_paragraph} : Cusanus leads us through a series of
reflections on seeing and on the face of God. God is located beyond both
imaginative exercise and conceptual understanding. \textbf{Nicholas
symbolizes our approach to this beyond by encouraging us to enter ``into
a certain secret and hidden silence wherein there is no knowledge or
concept of a face,'' characterizing it as an ``obscuring mist, haze,
darkness or ignorance.'' He invokes the coincidence of opposites from On
Learned Ignorance and proposes his second central metaphor: the wall of
paradise.}
\end{quote}

\begin{quote}
\emph{avg\_grammar\_rating} : 3.3\\
\emph{avg\_answerability\_rating} : 3.7\\
\emph{sum\_yes\_meaningful} : 2\\
\emph{sum\_no\_meaning} : 1\\
\emph{sum\_maybe\_meaning} : 0
\end{quote}

\hypertarget{what-kind-of-knowledge-is-our-knowledge-derived-from}{%
\subsection{What kind of knowledge is our knowledge derived
from?}\label{what-kind-of-knowledge-is-our-knowledge-derived-from}}

\begin{quote}
\emph{summarized\_paragraph} : Nicholas never questioned that the varied
things we discover in the natural universe and fashion ourselves in the
social and cultural milieu exist independently of our minds. \textbf{But
the question here is whether our knowledge is derived from what is
independent of mind or is in whole or part the result of linguistic and
conceptual measures we learn, construct and employ in dealing with
reality.} If knowing is creative or productive, solely a matter of our
``measuring,'' it is easy to see how it is an image of God's creating,
but not how it's a likening to extra-mental things.
\end{quote}

\begin{quote}
\emph{avg\_grammar\_rating} : 4.3\\
\emph{avg\_answerability\_rating} : 4.3\\
\emph{sum\_yes\_meaningful} : 3\\
\emph{sum\_no\_meaning} : 0\\
\emph{sum\_maybe\_meaning} : 0
\end{quote}

\hypertarget{what-is-our-knowledge-derived-from}{%
\subsection{What is our knowledge derived
from?}\label{what-is-our-knowledge-derived-from}}

\begin{quote}
\emph{summarized\_paragraph} : Nicholas never questioned that the varied
things we discover in the natural universe and fashion ourselves in the
social and cultural milieu exist independently of our minds. \textbf{But
the question here is whether our knowledge is derived from what is
independent of mind or is in whole or part the result of linguistic and
conceptual measures we learn, construct and employ in dealing with
reality.} If knowing is creative or productive, solely a matter of our
``measuring,'' it is easy to see how it is an image of God's creating,
but not how it's a likening to extra-mental things.
\end{quote}

\begin{quote}
\emph{avg\_grammar\_rating} : 4.7\\
\emph{avg\_answerability\_rating} : 4.0\\
\emph{sum\_yes\_meaningful} : 3\\
\emph{sum\_no\_meaning} : 0\\
\emph{sum\_maybe\_meaning} : 0
\end{quote}

\hypertarget{what-type-of-things-are-the-opposites-of-the-god}{%
\subsection{What type of things are the opposites of the
god?}\label{what-type-of-things-are-the-opposites-of-the-god}}

\begin{quote}
\emph{summarized\_paragraph} : Cusanus says we can do more justice to
the unique relation between God and creatures. \textbf{He identifies the
infinite God with this ``oppositeness of opposites.'' Now the
`opposites' in question are the ordinary things of our experience that
are separate and distinct and that may have opposed or mutually
exclusive properties.} Cusanus proposes some possible indirect routes
that will give us no positive insight or conceptual grasp of the divine
Essence. For instance, if we look to the very oppositions and
contradictions that plague our normal thinking about God.
\end{quote}

\begin{quote}
\emph{avg\_grammar\_rating} : 5.0\\
\emph{avg\_answerability\_rating} : 4.0\\
\emph{sum\_yes\_meaningful} : 4\\
\emph{sum\_no\_meaning} : 0\\
\emph{sum\_maybe\_meaning} : 0
\end{quote}

\hypertarget{what-is-the-word-for-a-mans-word-for-imagination}{%
\subsection{What is the word for a man's word for
imagination?}\label{what-is-the-word-for-a-mans-word-for-imagination}}

\begin{quote}
\emph{summarized\_paragraph} : It is the mind's power to discriminate
and make sense of what we perceive, imagine or remember that Cusanus
emphasizes. He writes, speaking of imagination, ``When sensible things
are not present, it {[}imagination{]} conforms itself to things in a
confused way and without discriminating one condition from another''
(Idiota de mente, c.7) The power of the mind to discriminate is the
power of imagination. It is the key to understanding the world around
us.
\end{quote}

\begin{quote}
\emph{avg\_grammar\_rating} : 2.3\\
\emph{avg\_answerability\_rating} : 2.0\\
\emph{sum\_yes\_meaningful} : 1\\
\emph{sum\_no\_meaning} : 2\\
\emph{sum\_maybe\_meaning} : 0
\end{quote}

\hypertarget{what-power-does-he-say-conforms-itself-to-things-in-a-confused-way}{%
\subsection{What power does he say conforms itself to things in a
confused
way?}\label{what-power-does-he-say-conforms-itself-to-things-in-a-confused-way}}

\begin{quote}
\emph{summarized\_paragraph} : It is the mind's power to discriminate
and make sense of what we perceive, imagine or remember that Cusanus
emphasizes. He writes, speaking of imagination, ``When sensible things
are not present, it {[}imagination{]} conforms itself to things in a
confused way and without discriminating one condition from another''
(Idiota de mente, c.7) The power of the mind to discriminate is the
power of imagination. It is the key to understanding the world around
us.
\end{quote}

\begin{quote}
\emph{avg\_grammar\_rating} : 4.7\\
\emph{avg\_answerability\_rating} : 4.3\\
\emph{sum\_yes\_meaningful} : 3\\
\emph{sum\_no\_meaning} : 0\\
\emph{sum\_maybe\_meaning} : 0
\end{quote}

\hypertarget{are-we-more-familiar-with-the-phrase-the-not-other-in-negative-ways-of-stating-self---identity}{%
\subsection{Are we more familiar with the phrase " the not other " in
negative ways of stating self -
identity?}\label{are-we-more-familiar-with-the-phrase-the-not-other-in-negative-ways-of-stating-self---identity}}

\begin{quote}
\emph{summarized\_paragraph} : In De li Non Aliud/On the Not Other
Nicholas returns to the ancient Platonic categories of the One and the
Other in order to re-construe in novel language what Christians believe
about the dependence of creatures on their Creator. Several of Cusanus'
later works use verbal coinages or Latin neologisms as ``names'' or
characterizations of God that are original with him, though they have
earlier echoes in Christian Neoplatonism. \textbf{In this dialogue he
uses the expression ``the not other'' as a substantive for God as the
divine Not-Other, even though we are more familiar with the phrase in
negative ways of stating self-identity.}
\end{quote}

\begin{quote}
\emph{avg\_grammar\_rating} : 5.0\\
\emph{avg\_answerability\_rating} : 5.0\\
\emph{sum\_yes\_meaningful} : 4\\
\emph{sum\_no\_meaning} : 0\\
\emph{sum\_maybe\_meaning} : 0
\end{quote}

\hypertarget{how-are-we-familiar-with-the-phrase-the-not-other-summarized_paragraph-in-de-li-non-aliudon-the-not-other-nicholas-returns-to-the-ancient-platonic-categories-of-the-one-and-the-other-in-order-to-re-construe-in-novel-language-what-christians-believe-about-the-dependence-of-creatures-on-their-creator.-several-of-cusanus-later-works-use-verbal-coinages-or-latin-neologisms-as-names-or-characterizations-of-god-that-are-original-with-him-though-they-have-earlier-echoes-in-christian-neoplatonism.-in-this-dialogue-he-uses-the-expression-the-not-other-as-a-substantive-for-god-as-the-divine-not-other-even-though-we-are-more-familiar-with-the-phrase-in-negative-ways-of-stating-self-identity.}{%
\subsection{\texorpdfstring{How are we familiar with the phrase " the
not other ``? \textgreater{} \emph{summarized\_paragraph} : In De li Non
Aliud/On the Not Other Nicholas returns to the ancient Platonic
categories of the One and the Other in order to re-construe in novel
language what Christians believe about the dependence of creatures on
their Creator. Several of Cusanus' later works use verbal coinages or
Latin neologisms as ``names'' or characterizations of God that are
original with him, though they have earlier echoes in Christian
Neoplatonism. \textbf{In this dialogue he uses the expression ``the not
other'' as a substantive for God as the divine Not-Other, even though we
are more familiar with the phrase in negative ways of stating
self-identity.}}{How are we familiar with the phrase " the not other ``? \textgreater{} summarized\_paragraph : In De li Non Aliud/On the Not Other Nicholas returns to the ancient Platonic categories of the One and the Other in order to re-construe in novel language what Christians believe about the dependence of creatures on their Creator. Several of Cusanus' later works use verbal coinages or Latin neologisms as ``names'' or characterizations of God that are original with him, though they have earlier echoes in Christian Neoplatonism. In this dialogue he uses the expression ``the not other'' as a substantive for God as the divine Not-Other, even though we are more familiar with the phrase in negative ways of stating self-identity.}}\label{how-are-we-familiar-with-the-phrase-the-not-other-summarized_paragraph-in-de-li-non-aliudon-the-not-other-nicholas-returns-to-the-ancient-platonic-categories-of-the-one-and-the-other-in-order-to-re-construe-in-novel-language-what-christians-believe-about-the-dependence-of-creatures-on-their-creator.-several-of-cusanus-later-works-use-verbal-coinages-or-latin-neologisms-as-names-or-characterizations-of-god-that-are-original-with-him-though-they-have-earlier-echoes-in-christian-neoplatonism.-in-this-dialogue-he-uses-the-expression-the-not-other-as-a-substantive-for-god-as-the-divine-not-other-even-though-we-are-more-familiar-with-the-phrase-in-negative-ways-of-stating-self-identity.}}

\begin{quote}
\emph{avg\_grammar\_rating} : 5.0\\
\emph{avg\_answerability\_rating} : 4.3\\
\emph{sum\_yes\_meaningful} : 2\\
\emph{sum\_no\_meaning} : 0\\
\emph{sum\_maybe\_meaning} : 1
\end{quote}

\hypertarget{what-is-god-different-from}{%
\subsection{What is God different
from?}\label{what-is-god-different-from}}

\begin{quote}
\emph{summarized\_paragraph} : God is not a dependent function of
creatures. \textbf{God is precisely not any of the others and so is not
other or different in the way creatures are.} Thinking God as Not-Other
requires a characteristic Cusan dialectical thinking, not simply
affirming or denying difference. We are to recognize and acknowledge
that the divine not-Other is both not one of the other and at once not
other than any or all of them. The ``Not'' in `` not-other''
differentiates God from creatures but does not exclude the divine Not-
other.
\end{quote}

\begin{quote}
\emph{avg\_grammar\_rating} : 5.0\\
\emph{avg\_answerability\_rating} : 4.3\\
\emph{sum\_yes\_meaningful} : 3\\
\emph{sum\_no\_meaning} : 0\\
\emph{sum\_maybe\_meaning} : 0
\end{quote}

\hypertarget{what-is-not-one-of-the-others}{%
\subsection{What is not one of the
others?}\label{what-is-not-one-of-the-others}}

\begin{quote}
\emph{summarized\_paragraph} : God is not a dependent function of
creatures. \textbf{God is precisely not any of the others and so is not
other or different in the way creatures are.} Thinking God as Not-Other
requires a characteristic Cusan dialectical thinking, not simply
affirming or denying difference. We are to recognize and acknowledge
that the divine not-Other is both not one of the other and at once not
other than any or all of them. The ``Not'' in `` not-other''
differentiates God from creatures but does not exclude the divine Not-
other.
\end{quote}

\begin{quote}
\emph{avg\_grammar\_rating} : 4.3\\
\emph{avg\_answerability\_rating} : 3.0\\
\emph{sum\_yes\_meaningful} : 3\\
\emph{sum\_no\_meaning} : 1\\
\emph{sum\_maybe\_meaning} : 0
\end{quote}

\hypertarget{what-does-nicholas-propose}{%
\subsection{What does nicholas
propose?}\label{what-does-nicholas-propose}}

\begin{quote}
\emph{summarized\_paragraph} : The natural universe, then, is the whole
or contracted maximum collectively constituted by the many beings in
space and time. \textbf{Nicholas proposes the quasi-Anaxagorean slogan
that ``each thing is in each thing: quodlibet in quolibet'' to emphasize
that the individual beings or parts are no less ``contracted'' images of
the whole created universe.} Just as God is present to each creature
that stands as a contracted image of the divine, so the universe as a
macrocosm ispresent to each creatures or constitutive part as microcosm.
\end{quote}

\begin{quote}
\emph{avg\_grammar\_rating} : 5.0\\
\emph{avg\_answerability\_rating} : 5.0\\
\emph{sum\_yes\_meaningful} : 3\\
\emph{sum\_no\_meaning} : 0\\
\emph{sum\_maybe\_meaning} : 0
\end{quote}

\hypertarget{what-is-the-universe}{%
\subsection{What is the universe?}\label{what-is-the-universe}}

\begin{quote}
\emph{summarized\_paragraph} : The natural universe, then, is the whole
or contracted maximum collectively constituted by the many beings in
space and time. \textbf{Nicholas proposes the quasi-Anaxagorean slogan
that ``each thing is in each thing: quodlibet in quolibet'' to emphasize
that the individual beings or parts are no less ``contracted'' images of
the whole created universe.} Just as God is present to each creature
that stands as a contracted image of the divine, so the universe as a
macrocosm ispresent to each creatures or constitutive part as microcosm.
\end{quote}

\begin{quote}
\emph{avg\_grammar\_rating} : 5.0\\
\emph{avg\_answerability\_rating} : 4.0\\
\emph{sum\_yes\_meaningful} : 4\\
\emph{sum\_no\_meaning} : 0\\
\emph{sum\_maybe\_meaning} : 0
\end{quote}

\hypertarget{what-kind-of-reality-are-mind---independent-things}{%
\subsection{What kind of reality are mind - independent
things?}\label{what-kind-of-reality-are-mind---independent-things}}

\begin{quote}
\emph{summarized\_paragraph} : \textbf{Nicholas avers that our knowledge
of the natural and cultural world embodied in and made systematic in the
technical and mechanical and liberal arts will remain ``conjectural''
The reason is that we are not dealing with the true reality of
mind-independent things whose true forms are one with God.} We only
encounter the physically located temporal realities that are images of
the really real. Only the concepts of mathematics are not conjectural
because we fashion or construct these ideas ourselves. Consequently our
conceptions of them can be precise and certain. As conceptual entities
they escape the sorts of change and bodily limits characteristic of the
physical world.
\end{quote}

\begin{quote}
\emph{avg\_grammar\_rating} : 4.7\\
\emph{avg\_answerability\_rating} : 4.0\\
\emph{sum\_yes\_meaningful} : 2\\
\emph{sum\_no\_meaning} : 1\\
\emph{sum\_maybe\_meaning} : 0
\end{quote}

\hypertarget{what-will-our-knowledge-of-the-natural-and-cultural-world-embodied-in-and-made-systematic-in-the-technical-and-mechanical-arts-remain}{%
\subsection{What will our knowledge of the natural and cultural world
embodied in and made systematic in the technical and mechanical arts
remain?}\label{what-will-our-knowledge-of-the-natural-and-cultural-world-embodied-in-and-made-systematic-in-the-technical-and-mechanical-arts-remain}}

\begin{quote}
\emph{summarized\_paragraph} : \textbf{Nicholas avers that our knowledge
of the natural and cultural world embodied in and made systematic in the
technical and mechanical and liberal arts will remain ``conjectural''
The reason is that we are not dealing with the true reality of
mind-independent things whose true forms are one with God.} We only
encounter the physically located temporal realities that are images of
the really real. Only the concepts of mathematics are not conjectural
because we fashion or construct these ideas ourselves. Consequently our
conceptions of them can be precise and certain. As conceptual entities
they escape the sorts of change and bodily limits characteristic of the
physical world.
\end{quote}

\begin{quote}
\emph{avg\_grammar\_rating} : 4.7\\
\emph{avg\_answerability\_rating} : 4.7\\
\emph{sum\_yes\_meaningful} : 3\\
\emph{sum\_no\_meaning} : 0\\
\emph{sum\_maybe\_meaning} : 0
\end{quote}

\hypertarget{what-kind-of-not---other-is-not-one-of-the-creatures}{%
\subsection{What kind of not - other is not one of the
creatures?}\label{what-kind-of-not---other-is-not-one-of-the-creatures}}

\begin{quote}
\emph{summarized\_paragraph} : \textbf{The divine Not-Other is not one
of the creatures, but in a different way than they are different from
one another.} Cusanus gives expression to this important difference
between finite and Infinite. Negatively, the Not- other is not finite as
the others are. Positively, the reflexivity characteristic of a limited
thing's self-identity also characterizes the not-Other's relation with
it. To put this more formally, the difference or opposition between
created things is both symmetrical and transitive.
\end{quote}

\begin{quote}
\emph{avg\_grammar\_rating} : 4.7\\
\emph{avg\_answerability\_rating} : 5.0\\
\emph{sum\_yes\_meaningful} : 3\\
\emph{sum\_no\_meaning} : 0\\
\emph{sum\_maybe\_meaning} : 0
\end{quote}

\hypertarget{the-divine-not---other-is-not-one-of-what}{%
\subsection{The divine not - other is not one of
what?}\label{the-divine-not---other-is-not-one-of-what}}

\begin{quote}
\emph{summarized\_paragraph} : \textbf{The divine Not-Other is not one
of the creatures, but in a different way than they are different from
one another.} Cusanus gives expression to this important difference
between finite and Infinite. Negatively, the Not- other is not finite as
the others are. Positively, the reflexivity characteristic of a limited
thing's self-identity also characterizes the not-Other's relation with
it. To put this more formally, the difference or opposition between
created things is both symmetrical and transitive.
\end{quote}

\begin{quote}
\emph{avg\_grammar\_rating} : 3.3\\
\emph{avg\_answerability\_rating} : 4.0\\
\emph{sum\_yes\_meaningful} : 2\\
\emph{sum\_no\_meaning} : 1\\
\emph{sum\_maybe\_meaning} : 1
\end{quote}

\hypertarget{what-is-one-way-we-can-speculate-about-the-non---finite-presence---in---absence}{%
\subsection{What is one way we can speculate about the non - finite
presence - in -
absence?}\label{what-is-one-way-we-can-speculate-about-the-non---finite-presence---in---absence}}

\begin{quote}
\emph{summarized\_paragraph} : \textbf{We are able to speculate about
the case of a non-finite presence-in-absence by starting with but moving
beyond the limits of the presence and absence we are familiar with in
the realm of limited things.} Now we turn to the presence of the
unfamiliar infinite One as what is finally required, even if not
obviously experienced, to keep the creaturely image present and real. In
this case, we may think that absence becomes what is metaphorical. But
God's presence is hardly like that of one physical thing to another. It
is ineluctably and literally necessary to explain the reality of
anything and everything.
\end{quote}

\begin{quote}
\emph{avg\_grammar\_rating} : 5.0\\
\emph{avg\_answerability\_rating} : 5.0\\
\emph{sum\_yes\_meaningful} : 2\\
\emph{sum\_no\_meaning} : 1\\
\emph{sum\_maybe\_meaning} : 0
\end{quote}

\hypertarget{what-cardinals-of-the-pope-has-to-be-conjectural}{%
\subsection{What cardinal's of the pope has to be
conjectural?}\label{what-cardinals-of-the-pope-has-to-be-conjectural}}

\begin{quote}
\emph{summarized\_paragraph} : \textbf{The cardinal's sight of the pope
has to be conjectural because anything extended can show, as it were,
but one side of itself to another embodied viewer.} Two other sources of
``otherness'' besides bodiliness underlie the limitations on perceptual
knowledge. These differ from both our mental capacities and what we are
looking at or listening to. This means the terms in which perceptual
judgments are expressed reflect the broader historical background and
interests of the perceiver as well as his or her linguistic community.
\end{quote}

\begin{quote}
\emph{avg\_grammar\_rating} : 4.0\\
\emph{avg\_answerability\_rating} : 5.0\\
\emph{sum\_yes\_meaningful} : 3\\
\emph{sum\_no\_meaning} : 0\\
\emph{sum\_maybe\_meaning} : 0
\end{quote}

\hypertarget{the-cardinals-sight-of-the-pope-has-to-be-conjectural-because-it-can-show-only-one-side-of-itself-to-another-embodied}{%
\subsection{The cardinal's sight of the pope has to be conjectural
because it can show only one side of itself to another
embodied?}\label{the-cardinals-sight-of-the-pope-has-to-be-conjectural-because-it-can-show-only-one-side-of-itself-to-another-embodied}}

\begin{quote}
\emph{summarized\_paragraph} : \textbf{The cardinal's sight of the pope
has to be conjectural because anything extended can show, as it were,
but one side of itself to another embodied viewer.} Two other sources of
``otherness'' besides bodiliness underlie the limitations on perceptual
knowledge. These differ from both our mental capacities and what we are
looking at or listening to. This means the terms in which perceptual
judgments are expressed reflect the broader historical background and
interests of the perceiver as well as his or her linguistic community.
\end{quote}

\begin{quote}
\emph{avg\_grammar\_rating} : 1.7\\
\emph{avg\_answerability\_rating} : 2.3\\
\emph{sum\_yes\_meaningful} : 0\\
\emph{sum\_no\_meaning} : 2\\
\emph{sum\_maybe\_meaning} : 1
\end{quote}

\hypertarget{what-is-a-kind-of-second---order-language-about-the-ways-in-which-we-are-forced-to-think-and-talk-about-divinity}{%
\subsection{What is a kind of second - order language about the ways in
which we are forced to think and talk about
divinity?}\label{what-is-a-kind-of-second---order-language-about-the-ways-in-which-we-are-forced-to-think-and-talk-about-divinity}}

\begin{quote}
\emph{summarized\_paragraph} : In negative and apophatic ``theology,''
we are not only told what God is not but led to reflect explicitly on
what God must be. \textbf{The result is a kind of second-order language
about the ways in which we are forced to think and talk about divinity.}
It is not that creatures coincide with God or God with creatures, but
that in God all else coincides as nothing else than God. At the same
time this ``coincidence'' underlines the divine Oneness that comprehends
all else.
\end{quote}

\begin{quote}
\emph{avg\_grammar\_rating} : 5.0\\
\emph{avg\_answerability\_rating} : 4.7\\
\emph{sum\_yes\_meaningful} : 3\\
\emph{sum\_no\_meaning} : 0\\
\emph{sum\_maybe\_meaning} : 0
\end{quote}

\hypertarget{what-is-not-one-of-the-other}{%
\subsection{What is not one of the
other?}\label{what-is-not-one-of-the-other}}

\begin{quote}
\emph{summarized\_paragraph} : God is not a dependent function of
creatures. God is precisely not any of the others and so is not other or
different in the way creatures are. Thinking God as Not-Other requires a
characteristic Cusan dialectical thinking, not simply affirming or
denying difference. \textbf{We are to recognize and acknowledge that the
divine not-Other is both not one of the other and at once not other than
any or all of them.} The ``Not'' in `` not-other'' differentiates God
from creatures but does not exclude the divine Not- other.
\end{quote}

\begin{quote}
\emph{avg\_grammar\_rating} : 4.3\\
\emph{avg\_answerability\_rating} : 3.3\\
\emph{sum\_yes\_meaningful} : 0\\
\emph{sum\_no\_meaning} : 2\\
\emph{sum\_maybe\_meaning} : 1
\end{quote}

\hypertarget{what-is-not-one-of-the-other-1}{%
\subsection{What is not one of the
other?}\label{what-is-not-one-of-the-other-1}}

\begin{quote}
\emph{summarized\_paragraph} : God is not a dependent function of
creatures. God is precisely not any of the others and so is not other or
different in the way creatures are. Thinking God as Not-Other requires a
characteristic Cusan dialectical thinking, not simply affirming or
denying difference. \textbf{We are to recognize and acknowledge that the
divine not-Other is both not one of the other and at once not other than
any or all of them.} The ``Not'' in `` not-other'' differentiates God
from creatures but does not exclude the divine Not- other.
\end{quote}

\begin{quote}
\emph{avg\_grammar\_rating} : 3.7\\
\emph{avg\_answerability\_rating} : 2.7\\
\emph{sum\_yes\_meaningful} : 0\\
\emph{sum\_no\_meaning} : 2\\
\emph{sum\_maybe\_meaning} : 1
\end{quote}
