\hypertarget{kurt-guxf6del}{%
\section{\texorpdfstring{\href{https://plato.stanford.edu/entries/goedel/index.html}{Kurt
Gödel}}{Kurt Gödel}}\label{kurt-guxf6del}}

\hypertarget{who-wrote-the-1947-book-what-is-cantors-continuum-problem}{%
\subsection{Who wrote the 1947 book " what is cantor's continuum
problem?
"}\label{who-wrote-the-1947-book-what-is-cantors-continuum-problem}}

\begin{quote}
\emph{summarized\_paragraph} : In his 1947 ``What is Cantor's Continuum
Problem?'', Gödel expounds the view that in the case of meaningful
propositions of mathematics, there is always a fact of the matter to be
decided in a yes or no fashion. If there exists a domain of mathematical
objects or concepts, then any meaningful proposition concerning them
must be either true or false. The Continuum Hypothesis is Gödel's
example of a meaningful question. The concept ``how many'' leads
``unambiguously'' to a definite meaning of the hypothesis.
\end{quote}

\begin{quote}
\emph{avg\_grammar\_rating} : nan\\
\emph{avg\_answerability\_rating} : nan\\
\emph{sum\_yes\_meaningful} : 0\\
\emph{sum\_no\_meaning} : 0\\
\emph{sum\_maybe\_meaning} : 0
\end{quote}

\hypertarget{what-type-of-universe-is-the-universe}{%
\subsection{What type of universe is the
universe?}\label{what-type-of-universe-is-the-universe}}

\begin{quote}
\emph{summarized\_paragraph} : Gödel was compelled to this view of L by
the Leibnizian{[}18{]} idea that, rather than the universe being
``small,'' that is, one with the minimum number of sets, it is more
natural to think of the set theoretic universe as being as large as
possible. This idea would be reflected in his interest in maximality
principles, i.e.~principles which are meant to capture the intuitive
idea that the universe of set theory is maximal in the sense that
nothing can be added.
\end{quote}

\begin{quote}
\emph{avg\_grammar\_rating} : nan\\
\emph{avg\_answerability\_rating} : nan\\
\emph{sum\_yes\_meaningful} : 0\\
\emph{sum\_no\_meaning} : 0\\
\emph{sum\_maybe\_meaning} : 0
\end{quote}

\hypertarget{who-gave-the-lecture-on-the-theory-of-relativism}{%
\subsection{Who gave the lecture on the theory of
relativism?}\label{who-gave-the-lecture-on-the-theory-of-relativism}}

\begin{quote}
\emph{summarized\_paragraph} : Gödel's earlier conception of rationalism
refers to mathematical rigor and includes the concept of having a
genuine proof. It is therefore a more radical one than that to which he
would later subscribe. One can see it at work at the end of the Gibbs
lecture, after a sequence of arguments in favor of realism are given.
\textbf{It can also be seen at the beginning of the Gödel Lectures on
the Theory of Relativism, in which he argues for the existence of a
``priceless'' proof.}
\end{quote}

\begin{quote}
\emph{avg\_grammar\_rating} : nan\\
\emph{avg\_answerability\_rating} : nan\\
\emph{sum\_yes\_meaningful} : 0\\
\emph{sum\_no\_meaning} : 0\\
\emph{sum\_maybe\_meaning} : 0
\end{quote}

\hypertarget{according-to-guxf6del-what-type-of-proof-does-he-believe-is-there}{%
\subsection{According to Gödel , what type of proof does he believe is
there?}\label{according-to-guxf6del-what-type-of-proof-does-he-believe-is-there}}

\begin{quote}
\emph{summarized\_paragraph} : Gödel's earlier conception of rationalism
refers to mathematical rigor and includes the concept of having a
genuine proof. It is therefore a more radical one than that to which he
would later subscribe. One can see it at work at the end of the Gibbs
lecture, after a sequence of arguments in favor of realism are given.
\textbf{It can also be seen at the beginning of the Gödel Lectures on
the Theory of Relativism, in which he argues for the existence of a
``priceless'' proof.}
\end{quote}

\begin{quote}
\emph{avg\_grammar\_rating} : nan\\
\emph{avg\_answerability\_rating} : nan\\
\emph{sum\_yes\_meaningful} : 0\\
\emph{sum\_no\_meaning} : 0\\
\emph{sum\_maybe\_meaning} : 0
\end{quote}

\hypertarget{what-did-bernays-write-that-the-result-of-guxf6dels-drew-the-attention-of-the-hilbert-school-to-two}{%
\subsection{What did bernays write that the result of Gödel's drew the
attention of the hilbert school to
two?}\label{what-did-bernays-write-that-the-result-of-guxf6dels-drew-the-attention-of-the-hilbert-school-to-two}}

\begin{quote}
\emph{summarized\_paragraph} : Gödel's result drew attention of the
Hilbert school to two observations: first, that intuitionistic logic
goes beyond finitism, and secondly, that finitist systems may not be the
only acceptable ones from the foundational point of view.
\textbf{Bernays has written that this result of Gödel's drew the
attention of the~Hilbert school to~two observations: ~that
intuitionistic~logic~goes
beyond~finitism~and~that~it~may~not~be~the~only~acceptable~system.}
\end{quote}

\begin{quote}
\emph{avg\_grammar\_rating} : nan\\
\emph{avg\_answerability\_rating} : nan\\
\emph{sum\_yes\_meaningful} : 0\\
\emph{sum\_no\_meaning} : 0\\
\emph{sum\_maybe\_meaning} : 0
\end{quote}

\hypertarget{what-did-bernays-write-that-the-result-of-guxf6dels-drew-the-attention-of-the-hilbert-school-to-two-1}{%
\subsection{What did bernays write that the result of Gödel's drew the
attention of the hilbert school to
two?}\label{what-did-bernays-write-that-the-result-of-guxf6dels-drew-the-attention-of-the-hilbert-school-to-two-1}}

\begin{quote}
\emph{summarized\_paragraph} : Gödel's result drew attention of the
Hilbert school to two observations: first, that intuitionistic logic
goes beyond finitism, and secondly, that finitist systems may not be the
only acceptable ones from the foundational point of view.
\textbf{Bernays has written that this result of Gödel's drew the
attention of the~Hilbert school to~two observations: ~that
intuitionistic~logic~goes
beyond~finitism~and~that~it~may~not~be~the~only~acceptable~system.}
\end{quote}

\begin{quote}
\emph{avg\_grammar\_rating} : nan\\
\emph{avg\_answerability\_rating} : nan\\
\emph{sum\_yes\_meaningful} : 0\\
\emph{sum\_no\_meaning} : 0\\
\emph{sum\_maybe\_meaning} : 0
\end{quote}

\hypertarget{what-can-be-used-to-interpret-the-former-by-means-of}{%
\subsection{What can be used to interpret the former by means
of?}\label{what-can-be-used-to-interpret-the-former-by-means-of}}

\begin{quote}
\emph{summarized\_paragraph} : Heyting arithmetic is defined to be the
same as classical arithmetic, except that the underlying predicate logic
is given by intuitionistic axioms and rules of inference. We now
consider Gödel 1933e, in which Gödel showed, in effect, that
intuitionistic arithmetic is only apparently weaker than classical
first-order arithmetic. \textbf{This is because the latter can be
interpreted within the former by means of a simple translation, and thus
to be convinced of the consistency of classical arithmetic it is enough
to be~convinced~of Heyting arithmetic.}
\end{quote}

\begin{quote}
\emph{avg\_grammar\_rating} : nan\\
\emph{avg\_answerability\_rating} : nan\\
\emph{sum\_yes\_meaningful} : 0\\
\emph{sum\_no\_meaning} : 0\\
\emph{sum\_maybe\_meaning} : 0
\end{quote}

\hypertarget{heyting-arithmetic-is-a-part-of-what-kind-of-arithmetic}{%
\subsection{Heyting arithmetic is a part of what kind of
arithmetic?}\label{heyting-arithmetic-is-a-part-of-what-kind-of-arithmetic}}

\begin{quote}
\emph{summarized\_paragraph} : Heyting arithmetic is defined to be the
same as classical arithmetic, except that the underlying predicate logic
is given by intuitionistic axioms and rules of inference. We now
consider Gödel 1933e, in which Gödel showed, in effect, that
intuitionistic arithmetic is only apparently weaker than classical
first-order arithmetic. \textbf{This is because the latter can be
interpreted within the former by means of a simple translation, and thus
to be convinced of the consistency of classical arithmetic it is enough
to be~convinced~of Heyting arithmetic.}
\end{quote}

\begin{quote}
\emph{avg\_grammar\_rating} : nan\\
\emph{avg\_answerability\_rating} : nan\\
\emph{sum\_yes\_meaningful} : 0\\
\emph{sum\_no\_meaning} : 0\\
\emph{sum\_maybe\_meaning} : 0
\end{quote}

\hypertarget{what-is-the-only-way-to-prove-a-theorem-without-a-formalization-of-it}{%
\subsection{What is the only way to prove a theorem without a
formalization of
it?}\label{what-is-the-only-way-to-prove-a-theorem-without-a-formalization-of-it}}

\begin{quote}
\emph{summarized\_paragraph} : The above proof of the Second
Incompleteness Theorem is deceptively simple as it avoids the
formalization. \textbf{A rigorous proof would have to establish the
proof of `if P ⊢ φ, then P ⋅ 0 ≠ 1' in P. The above proof is not the
only way to prove the Theorem, but it is the best way to show that it
can be proved without a formalization of the proof.} The proof is
published in the book `Incompletes', published by Oxford University
Press.
\end{quote}

\begin{quote}
\emph{avg\_grammar\_rating} : nan\\
\emph{avg\_answerability\_rating} : nan\\
\emph{sum\_yes\_meaningful} : 0\\
\emph{sum\_no\_meaning} : 0\\
\emph{sum\_maybe\_meaning} : 0
\end{quote}

\hypertarget{a-rigorous-proof-would-have-to-do-what}{%
\subsection{A rigorous proof would have to do
what?}\label{a-rigorous-proof-would-have-to-do-what}}

\begin{quote}
\emph{summarized\_paragraph} : The above proof of the Second
Incompleteness Theorem is deceptively simple as it avoids the
formalization. \textbf{A rigorous proof would have to establish the
proof of `if P ⊢ φ, then P ⋅ 0 ≠ 1' in P. The above proof is not the
only way to prove the Theorem, but it is the best way to show that it
can be proved without a formalization of the proof.} The proof is
published in the book `Incompletes', published by Oxford University
Press.
\end{quote}

\begin{quote}
\emph{avg\_grammar\_rating} : nan\\
\emph{avg\_answerability\_rating} : nan\\
\emph{sum\_yes\_meaningful} : 0\\
\emph{sum\_no\_meaning} : 0\\
\emph{sum\_maybe\_meaning} : 0
\end{quote}

\hypertarget{what-is-the-difference-between-the-concept-of-provability-in-a-specified-formal-system-and-provability-by-any-correct-means}{%
\subsection{What is the difference between the concept of provability in
a specified formal system and provability by any correct
means?}\label{what-is-the-difference-between-the-concept-of-provability-in-a-specified-formal-system-and-provability-by-any-correct-means}}

\begin{quote}
\emph{summarized\_paragraph} : Gödel's result marks the beginning of
provability logic. \textbf{It makes exact the difference between the
concept of ``provability in a specified formal system' and that of
`provability by any correct means' This result of Gödel's , which marks
the Beginning of Provability logic, is known as the Gödel paradox.} It
is the result of the experiment of proving the existence of a certain
type of truth in a formal system. It is also known as~the Gödel
conjecture.
\end{quote}

\begin{quote}
\emph{avg\_grammar\_rating} : nan\\
\emph{avg\_answerability\_rating} : nan\\
\emph{sum\_yes\_meaningful} : 0\\
\emph{sum\_no\_meaning} : 0\\
\emph{sum\_maybe\_meaning} : 0
\end{quote}

\hypertarget{what-is-still-an-element-of-n}{%
\subsection{What is still an element of \textless{} n , ∈
\textgreater?}\label{what-is-still-an-element-of-n}}

\begin{quote}
\emph{summarized\_paragraph} : \textbf{We used the Mostowski Collapse to
construct the transitive set N. We now return to the proof of the CH in
L. As it turns out, the real number A is still an element of \textless{}
N, ∈ \textgreater{} .} By basic properties of L, A must be \textless{}
Lα , ℵ0 \textgreater{} for some α. Since N is countable, α is also
countable too. Thus A is constructible on a countable level, which was
to have been shown.
\end{quote}

\begin{quote}
\emph{avg\_grammar\_rating} : nan\\
\emph{avg\_answerability\_rating} : nan\\
\emph{sum\_yes\_meaningful} : 0\\
\emph{sum\_no\_meaning} : 0\\
\emph{sum\_maybe\_meaning} : 0
\end{quote}

\hypertarget{whos-result-drew-attention-of-the-hilbert-school-to-two-observations}{%
\subsection{Who's result drew attention of the hilbert school to two
observations?}\label{whos-result-drew-attention-of-the-hilbert-school-to-two-observations}}

\begin{quote}
\emph{summarized\_paragraph} : \textbf{Gödel's result drew attention of
the Hilbert school to two observations: first, that intuitionistic logic
goes beyond finitism, and secondly, that finitist systems may not be the
only acceptable ones from the foundational point of view.} Bernays has
written that this result of Gödel's drew the attention of the~Hilbert
school to~two observations: ~that intuitionistic~logic~goes
beyond~finitism~and~that~it~may~not~be~the~only~acceptable~system.
\end{quote}

\begin{quote}
\emph{avg\_grammar\_rating} : nan\\
\emph{avg\_answerability\_rating} : nan\\
\emph{sum\_yes\_meaningful} : 0\\
\emph{sum\_no\_meaning} : 0\\
\emph{sum\_maybe\_meaning} : 0
\end{quote}

\hypertarget{what-does-it-mean-to-better-understand-v}{%
\subsection{What does it mean to better understand
v?}\label{what-does-it-mean-to-better-understand-v}}

\begin{quote}
\emph{summarized\_paragraph} : Study of forcing leads to a better
understanding of V, says Hugh Woodin. Study of models of a theory is
useful to understand the theory itself, he says. \textbf{This is
reminiscent of a remark of Hugh Wood in, that studying forcing leads~to
a better Understanding of V. The general principle being that studying
the models is~useful~to obtain a better picture of V , says Woodin~in
his book, ``V: The Theory of Forces and Forces in the World''}
\end{quote}

\begin{quote}
\emph{avg\_grammar\_rating} : nan\\
\emph{avg\_answerability\_rating} : nan\\
\emph{sum\_yes\_meaningful} : 0\\
\emph{sum\_no\_meaning} : 0\\
\emph{sum\_maybe\_meaning} : 0
\end{quote}

\hypertarget{what-does-it-mean-to-better-understand-v-1}{%
\subsection{What does it mean to better understand
v?}\label{what-does-it-mean-to-better-understand-v-1}}

\begin{quote}
\emph{summarized\_paragraph} : Study of forcing leads to a better
understanding of V, says Hugh Woodin. Study of models of a theory is
useful to understand the theory itself, he says. \textbf{This is
reminiscent of a remark of Hugh Wood in, that studying forcing leads~to
a better Understanding of V. The general principle being that studying
the models is~useful~to obtain a better picture of V , says Woodin~in
his book, ``V: The Theory of Forces and Forces in the World''}
\end{quote}

\begin{quote}
\emph{avg\_grammar\_rating} : nan\\
\emph{avg\_answerability\_rating} : nan\\
\emph{sum\_yes\_meaningful} : 0\\
\emph{sum\_no\_meaning} : 0\\
\emph{sum\_maybe\_meaning} : 0
\end{quote}

\hypertarget{what-kind-of-foundation-do-many-mathematicians-think-the-axioms-of-set-theory-provide}{%
\subsection{What kind of foundation do many mathematicians think the
axioms of set theory
provide?}\label{what-kind-of-foundation-do-many-mathematicians-think-the-axioms-of-set-theory-provide}}

\begin{quote}
\emph{summarized\_paragraph} : **``I believed that it was so clear that
axiomatization in terms of sets was not a satisfactory ultimate
foundation of mathematics that mathematicians would, for the most part,
not be very much concerned with it'' ``But in recent times I have seen
to my surprise that so many mathematicians think that these axioms of
set theory provide the ideal foundation for mathematics'' ``It seemed to
me that the time had come to publish a critique.** \ldots{} first, I
have in the meantime been occupied with other problems''
\end{quote}

\begin{quote}
\emph{avg\_grammar\_rating} : nan\\
\emph{avg\_answerability\_rating} : nan\\
\emph{sum\_yes\_meaningful} : 0\\
\emph{sum\_no\_meaning} : 0\\
\emph{sum\_maybe\_meaning} : 0
\end{quote}

\hypertarget{what-does-one-confines-oneself-to}{%
\subsection{What does one confines oneself
to?}\label{what-does-one-confines-oneself-to}}

\begin{quote}
\emph{summarized\_paragraph} : Perhaps the reason why no progress is
made in mathematics (and there are so many unsolved problems), is that
one confines oneself to the ext{[}ensional{]}---thence also the feeling
of disappointment in the case of many theories, e.g., propositional
logic and formalisation altogether. {[}22{]} {[}22{]}' Perhaps the
reasonwhy no progress has been made in Mathematics is that~one confines
oneself~to the ext {[}ensional{]}~- thence also~the feeling
of~disappointment~in~the case of~propositional~logic.
\end{quote}

\begin{quote}
\emph{avg\_grammar\_rating} : nan\\
\emph{avg\_answerability\_rating} : nan\\
\emph{sum\_yes\_meaningful} : 0\\
\emph{sum\_no\_meaning} : 0\\
\emph{sum\_maybe\_meaning} : 0
\end{quote}

\hypertarget{what-axioms-assert-to-exist}{%
\subsection{What axioms assert to
exist?}\label{what-axioms-assert-to-exist}}

\begin{quote}
\emph{summarized\_paragraph} : The continuum problem is solved by
finding an enumeration of the reals which is indexed by the countable
ordinals. \textbf{The intuition behind the proof is to build a ``small''
model, one in which the absolute minimum number of reals is allowed,
while at the same time the model is large enough to be closed under all
the operations the ZF axioms assert to exist.} The strategy had been
recognized as a promising one already by ~Hilbert~and was used to solve
the continuum problem.
\end{quote}

\begin{quote}
\emph{avg\_grammar\_rating} : nan\\
\emph{avg\_answerability\_rating} : nan\\
\emph{sum\_yes\_meaningful} : 0\\
\emph{sum\_no\_meaning} : 0\\
\emph{sum\_maybe\_meaning} : 0
\end{quote}

\hypertarget{what-was-set-theorys-1923-result}{%
\subsection{What was set theory's 1923
result?}\label{what-was-set-theorys-1923-result}}

\begin{quote}
\emph{summarized\_paragraph} : The failure of categoricity was already
taken note of by Skolem in 1923, because it follows from the
Löwenheim-Skolem Theorem. Any first order theory in a countable language
that has a model has acountable model. \textbf{As for set theory, the
failure of set theory was already take note of in 1923.} Theorem: A
theory with a model is a theory with countable models. Theorems: Theorem
1: Any theory with model has model; Theorem 2: A model has countable
neighbors.
\end{quote}

\begin{quote}
\emph{avg\_grammar\_rating} : nan\\
\emph{avg\_answerability\_rating} : nan\\
\emph{sum\_yes\_meaningful} : 0\\
\emph{sum\_no\_meaning} : 0\\
\emph{sum\_maybe\_meaning} : 0
\end{quote}

\hypertarget{what-theory-was-not-taken-note-of-in-1923}{%
\subsection{What theory was not taken note of in
1923?}\label{what-theory-was-not-taken-note-of-in-1923}}

\begin{quote}
\emph{summarized\_paragraph} : The failure of categoricity was already
taken note of by Skolem in 1923, because it follows from the
Löwenheim-Skolem Theorem. Any first order theory in a countable language
that has a model has acountable model. \textbf{As for set theory, the
failure of set theory was already take note of in 1923.} Theorem: A
theory with a model is a theory with countable models. Theorems: Theorem
1: Any theory with model has model; Theorem 2: A model has countable
neighbors.
\end{quote}

\begin{quote}
\emph{avg\_grammar\_rating} : nan\\
\emph{avg\_answerability\_rating} : nan\\
\emph{sum\_yes\_meaningful} : 0\\
\emph{sum\_no\_meaning} : 0\\
\emph{sum\_maybe\_meaning} : 0
\end{quote}

\hypertarget{hilbert-and-ackermann-ask-whether-a-given-axiom-system-for-what-order-is-complete}{%
\subsection{Hilbert and ackermann ask whether a given axiom system for
what order is
complete?}\label{hilbert-and-ackermann-ask-whether-a-given-axiom-system-for-what-order-is-complete}}

\begin{quote}
\emph{summarized\_paragraph} : \textbf{Hilbert and Ackermann ask whether
a certain explicitly given axiom system for the first order predicate
calculus ``\ldots is complete in the sense that from it all logical
formulas that are correct for each domain of individuals can be
derived\ldots{}'' The question is whether there is a system that is
``complete'' in the way that ``all logical formulas'' are derived from
it.} The answer is yes, they say, but it is not clear if there is such a
system.
\end{quote}

\begin{quote}
\emph{avg\_grammar\_rating} : nan\\
\emph{avg\_answerability\_rating} : nan\\
\emph{sum\_yes\_meaningful} : 0\\
\emph{sum\_no\_meaning} : 0\\
\emph{sum\_maybe\_meaning} : 0
\end{quote}

\hypertarget{what-does-the-word-natural-mean}{%
\subsection{What does the word natural
mean?}\label{what-does-the-word-natural-mean}}

\begin{quote}
\emph{summarized\_paragraph} : \textbf{Gödel may have meant by ``natural
completion'' here ``the correct completion,'' or he may have mean to say
no more than that the Axiom of Constructibility determines the notion of
set in a definite way.} In any case he used the term `natural'
differently in a conversation with Wang on constructibility in 1972. He
may have used the word ` natural' to refer to the `correct completion'
of a set.
\end{quote}

\begin{quote}
\emph{avg\_grammar\_rating} : nan\\
\emph{avg\_answerability\_rating} : nan\\
\emph{sum\_yes\_meaningful} : 0\\
\emph{sum\_no\_meaning} : 0\\
\emph{sum\_maybe\_meaning} : 0
\end{quote}

\hypertarget{what-should-be-decided-once-conceptual-analysis-has-been-carried-out}{%
\subsection{What should be decided once conceptual analysis has been
carried
out?}\label{what-should-be-decided-once-conceptual-analysis-has-been-carried-out}}

\begin{quote}
\emph{summarized\_paragraph} : Gödel believed that once the appropriate
methods have been developed, philosophical problems such as those in
ethics can be decisively solved. \textbf{As for mathematical assertions,
such as the Continuum Hypothesis in set theory, once conceptual analysis
has been carried out in the right way, that is, once the basic concepts,
like that of ``set,'' have been completely clarified, the Continuity
Hypotheses should be able to be decided, he says.} He adds: ``Formal
rights comprise a real science.''
\end{quote}

\begin{quote}
\emph{avg\_grammar\_rating} : nan\\
\emph{avg\_answerability\_rating} : nan\\
\emph{sum\_yes\_meaningful} : 0\\
\emph{sum\_no\_meaning} : 0\\
\emph{sum\_maybe\_meaning} : 0
\end{quote}

\hypertarget{what-does-a-real-number-thought-of-as-a-set-of}{%
\subsection{What does a real number thought of as a set
of?}\label{what-does-a-real-number-thought-of-as-a-set-of}}

\begin{quote}
\emph{summarized\_paragraph} : Every real is constructed already on a
countable level of the L-hierarchy. \textbf{To show this Gödel argued as
follows: Suppose A is a real number thought of as a set of natural
numbers.} By a simple procedure A can be converted into a transitive
model. This procedure, used by Gödel already in 1937, was explicitly
isolated by Mostowski. The resulting model is referred to as the
Mostowski Collapse. The difficulty here, if not of the whole proof
altogether, lies in showing that every real is constructible.
\end{quote}

\begin{quote}
\emph{avg\_grammar\_rating} : nan\\
\emph{avg\_answerability\_rating} : nan\\
\emph{sum\_yes\_meaningful} : 0\\
\emph{sum\_no\_meaning} : 0\\
\emph{sum\_maybe\_meaning} : 0
\end{quote}

\hypertarget{what-is-a-real-number-thought-of-as}{%
\subsection{What is a real number thought of
as?}\label{what-is-a-real-number-thought-of-as}}

\begin{quote}
\emph{summarized\_paragraph} : Every real is constructed already on a
countable level of the L-hierarchy. \textbf{To show this Gödel argued as
follows: Suppose A is a real number thought of as a set of natural
numbers.} By a simple procedure A can be converted into a transitive
model. This procedure, used by Gödel already in 1937, was explicitly
isolated by Mostowski. The resulting model is referred to as the
Mostowski Collapse. The difficulty here, if not of the whole proof
altogether, lies in showing that every real is constructible.
\end{quote}

\begin{quote}
\emph{avg\_grammar\_rating} : nan\\
\emph{avg\_answerability\_rating} : nan\\
\emph{sum\_yes\_meaningful} : 0\\
\emph{sum\_no\_meaning} : 0\\
\emph{sum\_maybe\_meaning} : 0
\end{quote}
